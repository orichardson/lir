\documentclass{article} % For LaTeX2e
\usepackage{iclr2026_conference,times}

\usepackage{hyperref}
\usepackage{url}


\usepackage{pdgtools}
\usepackage[nobox]{restatelinks}

%% theorems %
    \usepackage{amsthm,thmtools}
    \theoremstyle{plain}
    % \newtheorem{theorem}{Theorem}[section]
    \newtheorem{theorem}{Theorem}
    \newtheorem{proposition}[theorem]{Proposition}
    \newtheorem{lemma}[theorem]{Lemma}
    \newtheorem{corollary}[theorem]{Corollary}
    \theoremstyle{definition}
    \newtheorem{definition}{Definition}
    \newtheorem{assumption}[definition]{Assumption}
    \theoremstyle{remark}
    \newtheorem{remark}[theorem]{Remark}


%%%%%%%%%%%%%% TODO CLEAN UP %%%%%%%%%%
    

    \usepackage[utf8]{inputenc} % allow utf-8 input
    \usepackage[T1]{fontenc}    % use 8-bit T1 fonts
    \usepackage{hyperref}       % hyperlinks
    \usepackage{url}            % simple URL typesetting
    \usepackage{booktabs}       % professional-quality tables
    \usepackage{amsfonts}       % blackboard math symbols
    \usepackage{nicefrac}       % compact symbols for 1/2, etc.
    \usepackage{microtype}      % microtypography
    \usepackage{xcolor}         % colors


    %%%%%%%%%%%%%%%%%%%%%%%%%%%%%%%%%%%%%%%
    %%%%%%%%%%% Custom  Preamble %%%%%%%%%%
    %%%%%%%%%%%%%%%%%%%%%%%%%%%%%%%%%%%%%%%
    \let\cite\citep
    \usepackage{amsmath,amssymb,mathtools}
    \usepackage{algorithm}
    \usepackage[noend]{algorithmic}
    \usepackage{enumitem}
    \usepackage{bbm}
    \usepackage[capitalize,noabbrev,nameinlink]{cleveref}


    \relax % Short arrows.
       \newcommand{\veryshortarrow}[1][3pt]{\mathrel{%
       \vcenter{\hbox{\rule[-.5\fontdimen8\textfont3]{#1}{\fontdimen8\textfont3}}}%
       \mkern-4mu\hbox{\usefont{U}{lasy}{m}{n}\symbol{41}}}}
       \makeatletter
       \setbox0\hbox{$\xdef\scriptratio{\strip@pt\dimexpr
       \numexpr(\sf@size*65536)/\f@size sp}$}
       \newcommand{\scriptveryshortarrow}[1][3pt]{\mathrel{%
       \vcenter{\hbox{\rule[0\fontdimen8\scriptfont3]
       {\scriptratio\dimexpr#1\relax}{\fontdimen8\scriptfont3}}}%
       \mkern-4mu\hbox{\let\f@size\sf@size\usefont{U}{lasy}{m}{n}\symbol{41}}}}
       % \newcommand{\sto}{\scriptveryshortarrow}
       \makeatother
       % \newcommand\sto{{{\scalebox{1}[1]{$\scriptstyle\to$}}}}
       %
       \newsavebox{\stosbox}
       \sbox{\stosbox}{\begin{tikzpicture}\draw[->] (0,0) to (0.15,0);\end{tikzpicture}}
       \newcommand\sto{\usebox\stosbox}
       %%%
       %%% hacky: now I have to do the same thing for green and blue
       \newsavebox{\stosboxgreen}
       \sbox{\stosboxgreen}{\begin{tikzpicture}\draw[->,green!70!black] (0,0) to (0.15,0);\end{tikzpicture}}
       \newcommand\stogreen{\usebox\stosboxgreen}
       \newsavebox{\stosboxblue}
       \sbox{\stosboxblue}{\begin{tikzpicture}\draw[->,blue] (0,0) to (0.15,0);\end{tikzpicture}}
       \newcommand\stoblue{\usebox\stosboxblue}
       
       % \newcommand\sto{\begin{tikzpicture}\draw[->] (0,0) to (0.15,0);\end{tikzpicture}}
       % \newcommand\sto{\mspace{1mu}\begin{tikzpicture}\draw[bend left=0,-,shorten <=0,shorten >=0,thin,solid,] (0,0) to (0.15,0);\end{tikzpicture}}
       % \usetikzlibrary{decorations.markings}
       % \newcommand\sto{\mspace{1mu}\begin{tikzpicture}
       %     \draw[bend left=0,-,shorten <=0,shorten >=0.2pt,thin,solid,decoration={markings,mark=at position 1 with {\arrow[scale=0.7]{>}}},
       %     postaction={decorate},] (0,0) to (0.1,0);\end{tikzpicture}}

    \let\Horig\H \let\H\relax
    \DeclareMathOperator{\H}{\mathrm{H}} % Entropy
    \DeclareMathOperator*{\Ex}{\mathbb{E}} % Expectation
    \DeclarePairedDelimiterX{\infdivx}[2]{(}{)}{#1\;\delimsize\|\;#2}
    \newcommand{\thickD}{I\mkern-8muD}
    \newcommand{\kldiv}{\thickD\infdivx}
    \newcommand\mat[1]{\mathbf{#1}}

    \newcommand\numberthis{\addtocounter{equation}{1}\tag{\theequation}}

    % \tikzset{factor/.style={draw,minimum width=1.5em, minimum height=1.5em,fill=black,text=white,font=\bfseries}}
    \tikzset{factor/.style={draw,minimum width=1.5em, minimum height=1.5em,fill=gray!50!black,text=white,font=\mathversion{bold}\bfseries}}


    \newcommand\btheta{\boldsymbol\theta}
    \newcommand\Ctx{\dg{C\mskip-2mut\mskip-2mux}}
    \newcommand\Mm{\dg{M}}
    \newcommand\MThetadense{\dg{M}(\mskip-1mu\Theta\mskip-1mu)}
    \newcommand\Attn{\dg{A\mskip-2.2mut\mskip-2mut\mskip-2mun}}
    \newcommand\Ctrl{\dg{C\mskip-2mut\mskip-2mur\mskip-2mul}}

    \newcommand\Msg{\dg{M\mskip-1mus\mskip-2mu g}}

    % \newcommand\attn{\mathit{Attn}}
    % \newcommand\ctrl{\mathit{Ctrl}}
    % \newcommand\attn{\mathit{A}}
    % \newcommand\ctrl{\mathit{C}}
    \newcommand\attn{A}
    \newcommand\ctrl{C}
    % \newcommand\btheta{\boldsymbol\theta}

    \hypersetup{colorlinks=true, linkcolor=blue!75!black, urlcolor=magenta, citecolor=green!50!black}
    \newcommand\vfull[1]{}
    \DeclareMathOperator*{\argmin}{arg\,min} % argmin
    \DeclareMathOperator*{\argmax}{arg\,max} % argmax


    % commenting
    \usepackage[linecolor=black!20,textsize=tiny]{todonotes}
    \setlength{\marginparwidth}{3.3cm}
    \newcommand{\mehran}[1]{\todo[backgroundcolor=blue!20]{\textbf{Mehran:} #1}}
    \newcommand{\ali}[1]{\todo[backgroundcolor=yellow!20]{\textbf{Ali:} #1}}
    \newcommand{\oliver}[1]{\todo[backgroundcolor=green!20]{\textbf{Oliver:} #1}}







\title{Local Inconsistency Resolution}

% Authors must not appear in the submitted version. They should be hidden
% as long as the \iclrfinalcopy macro remains commented out below.
% Non-anonymous submissions will be rejected without review.
\author{
% Antiquus S.~Hippocampus, Natalia Cerebro \& Amelie P. Amygdale \thanks{ Use footnote for providing further information
% about author (webpage, alternative address)---\emph{not} for acknowledging
% funding agencies.  Funding acknowledgements go at the end of the paper.} \\
% Department of Computer Science\\
% Cranberry-Lemon University\\
% Pittsburgh, PA 15213, USA \\
% \texttt{\{hippo,brain,jen\}@cs.cranberry-lemon.edu} \\
% \And
% Ji Q. Ren \& Yevgeny LeNet \\
% Department of Computational Neuroscience \\
% University of the Witwatersrand \\
% Joburg, South Africa \\
% \texttt{\{robot,net\}@wits.ac.za} \\
% \AND
% Coauthor \\
% Affiliation \\
% Address \\
% \texttt{email}
% Oliver E.~Richardson, 
    % TODO: determine author order
    % Abdessamad 
    % Mandana
    % Mehran
    % Ali
    % Yoshua
AUTHORS \\
Department of Computer Science\\
University of Montreal / Mila\\
Montr\'{e}al, QC H3T 1J4 \\
\texttt{\{-,-,-\}@mila.quebec}
}

% The \author macro works with any number of authors. There are two commands
% used to separate the names and addresses of multiple authors: \And and \AND.
%
% Using \And between authors leaves it to \LaTeX{} to determine where to break
% the lines. Using \AND forces a linebreak at that point. So, if \LaTeX{}
% puts 3 of 4 authors names on the first line, and the last on the second
% line, try using \AND instead of \And before the third author name.

\newcommand{\fix}{\marginpar{FIX}}
\newcommand{\new}{\marginpar{NEW}}


% \iclrfinalcopy % Uncomment for camera-ready version, but NOT for submission.

\begin{document}

\maketitle

\begin{abstract}
    %%% Ali's draft
    %Building on Probabilistic Dependency Graphs (PDGs)---a new class of directed graphical models capable of representing inconsistent beliefs and generalizing Bayesian Networks---we introduce a generic algorithm for learning and approximate inference across a wide class of probabilistic models. Our algorithm is based on a simple epistemic principle: focus on a subset of variables and iteratively resolve inconsistencies using the parameters under control. This framework naturally unifies and generalizes a variety of existing methods, including the Expectation-Maximization (EM) algorithm, belief propagation, adversarial training, GANs, and GFlowNets. PDGs provide the underlying representational foundation, offering modularity, flexibility in incorporating new information, and a scoring function formalism that quantifies model-data incompatibility. Furthermore, PDGs generalize to  Bayesian Networks, factor graphs, and exponential family models, highlighting their expressive power in a broad range of machine learning scenarios.
    %%% Original draft
    % We present a generic algorithm for learning and approximate inference across a broad class of probabilistic models, unifying many important algorithms in the literature (e.g., the EM algorithm, adversarial training, belief propagation, the training of GANs and GFlowNets). Each algorithm  can be naturally viewed as an instance of a simple procedure that with an intuitive epistemic interpretation: focus your attention on a subset of variables and resolve the inconsistencies among them using the parameters under control.  Our account is based on the theory of probabilistic dependency graphs (PDGs), an expressive class of graphical models rooted in information theory, which can capture inconsistent beliefs and many scenarios in machine learning.  
    %%% New abstract, 
    % Building on Probabilistic Dependency Graphs (PDGs)---a class of directed graphical models capable of representing inconsistent beliefs---
    we present a generic algorithm for learning and approximate inference across a wide class of probabilistic models. Our algorithm is based on a simple epistemic principle: focus on a subset of variables and iteratively resolve inconsistencies using the parameters under control. This framework naturally unifies and generalizes a variety of existing methods, including the Expectation-Maximization (EM) algorithm, belief propagation, adversarial training, GANs, and GFlowNets. PDGs provide the underlying representational foundation, offering modularity, flexibility in incorporating new information, and a scoring function formalism that quantifies model-data incompatibility. Furthermore, PDGs generalize to  Bayesian Networks, factor graphs, and exponential family models, highlighting their expressive power in a broad range of machine learning scenarios.\end{abstract}


\section{Introduction}

% Probabilistic models form the cornerstone of modern machine learning, providing structured ways to represent uncertainty and belief. However, real-world information is often inherently inconsistent, reflecting collecting beliefs or contradictory data. Addressing these inconsistencies systematically can enhance model reliability and generalization. 
%
%
What causes one to change their mind? To learn, act, and draw inferences? 
%Mandana's new intro material
From a cognitive and neurosientific perspective, such mental operation often stem from detection and resolution of internal inconsistencies. The brain continuously reconciles sensory inputs, prior beliefs, and predictions—an operation though to underlie learning, perception, and planning. In predictive coding theories of the brain, for instance, discrepancies between expected and actual sensory inputs—known as prediction errors—are treated as inconsistency signals that drive learning via local adjustments to neural representation \todo{cite Friston 2005; Rao \& Ballard 1999}.

Inspired by this  neurocognitive principle, we explore a computational framework for local insonsistency resolution (LIR) in probabilistic models.. The key idea is simple: rather than resolving all inconsistencies in a belief system globally (which may be intractable), we iteratively identify and locally reduce inconsistency over a subset of variables using only parameters under direct control. This mirrors how both human cognition and neural systems operate under limited attention and bounded control, often correcting inconsistencies in one area while remaining unaware of or unable to act on others\todo{cite}.
We argue that all of these mental operations can be viewed as 
attempts to mitigate or resolve internal inconsistency between one's beliefs. Sometimes this the result of discovering new information that contradicts our beliefs;
other times it comes from new awareness of discrepancies between beliefs we already hold.
%
% The process is often a gradual one: inconsistencies are dealt with incrementally, and one at a time.
% However, people do not change their minds all at once, keeping in mind everything they know;
% However, it is impossible to respond to this inconsistency all at once, keeping in mind everything you know;
% Indeed, we can only respond to the ones we are aware of, and inconsistencies can be difficult to detect.
Inconsistency can be difficult to detect, however, and 
   % can only be resolved once we are aware of it.
   it is difficult to see how an inconsistency 
       could be resolved (without massive collateral damage to one's belief state)
       without becoming aware of it.
% Further complicating things,
% Typically
% Further complicating our lives, some parts of the picture are beyond our control;
A further complication is that some things may be beyond our control;
   for example, we might receive conflicting information
       from two trusted sources and be unable to  resolve
       their disagreement.
   % impossible to respond to this inconsistency all at once, keeping in mind everything you know;
% So in practice, we resolve inconsistencies \emph{locallly}---little by little, and one at a time.
So in practice, we resolve inconsistencies \emph{locallly}---%
   by looking at only a small part of the picture,
   % and changing an even smaller part.
   and changing another part of it.
   % ---little by little, and one at a time.
This can have externalities; fixing one inconsistency can easily create others out of view.


% Nevertheless, this process of locally resolving inconsistency can be quite useful.
% Nevertheless,
Despite its imperfections,
   this process of locally resolving inconsistency can be quite useful.
   % And, as we have formalized it in the probabilistic setting,
   As we shall soon see,
   it is a powerful recipe for learning and approximate inference.
% In fact, we will show that it naturally captures a broad class of
% We provide a formalization of it, which we use to show how
%
% We provide a formalization of this process, and show
% We formalize the intuition above
% We provide a formalization of this process
We formalize the process
   in the language of probability and
   convex optimization,
   and show how that many popular techniques in the literature
   % can be viewed as local inconsistency resolution.
   arise naturally as instances of it.
   % Finally, we propse a new algorithm
% We argue that this intuition applies not only to people, but also to a
   % broad class of AI systems from the last 20 years.
% This paper describes a general process by which this occurs,

% Historically, much of the belief revision literature has focused on
% conflicts between propositions and entrenchment \cite{agm,...}.


Our approach leans heavily on the theory of
Probabilistic Dependency Graphs (PDGs), which are very flexible graphical
   models that allow for arbitrary---even inconsistent---%
   probabilistic information, weighted by confidence \cite{pdg-aaai}.
% In addition, PDG semantics
There is a natural way
% information-theoretic way
   % of measuring the inconsistency of a PDG,
   to measure how inconsistent a PDG is,
   and many standard loss functions
   can be viewed as measuring the inconsistency of a PDG that
   describes the appropriate situation \cite{one-true-loss}.
\vfull{
   Recently, techniques have been developed to calculate this inconsistency
   in polynomial time for bounded tree-width,
   although it scales exponentially with the tree-width of the graph \cite{pdg-infer}.
   As we move to variables that are continuous variables of large dimension,
       it becomes intractable to calculate this global inconsistency
       even for small graphs---the log evidence
       of a latent variable model \cite{one-true-loss} can be
       represented as a PDG inconsistency, for example.
}
% Our work
% We provide an algorithm that
%
  % implicit when we regard inconsistency as a loss function,
  % and generalizing it to training procedures other than .
  % allowing us to describe details about the training process other than
  % ``somehow minimize this global loss''.
%
% It can be intractable to calculate for .
% Calculating the inconsistency of a PDG can be intractable.
% Calculating a PDG's inconsistency can be intractable.
% Our algorithm
We introduce an algorithm
   to operationalize
   the process of adjusting parameters
   to resolve this inconsistency.

% Calculating a PDG's inconsistency can be intractable.
% In general, calculating a PDG's degree of inconsistency can be intractable.
Computing a PDG's degree of inconsistency can be straightforward in some cases, but it is NP-hard in general \citep{pdg-infer}.
% \citet{one-true-loss} shows how variational inference can
%     be understood as the adoption of extra beliefs to
%     % overapproximate the overall inconsistency in a way that is easier to calculate.
%     % in a tractable way.
%     get an overapproximation that is easier to calculate.
Variational inference \citep{blei2017variational,kingma2013autoencoding,jordan1999introduction} can be understood as
   the practice adopting extra beliefs to
   % overapproximate the overall inconsistency in a way that is easier to calculate.
   % in a tractable way.
   % overapproximate it in a way that is easier to calculate \cite{one-true-loss}.
   get an over\-approximation of inconsistency that is easier to calculate \cite{one-true-loss}.
% Our algorithm allows for this kind of
Our approach also enables the opposite:
% This is one use of our algorithm,   but we also enable the opposite:
   focusing on small parts of the graph at a time to
    % get
   address tractable under\-approximations of the global inconsistency.
% We show how this
% We show how this technique specializes and suggests a
   % which ideally stil makes progress.
% This has additional for parallelization.
% Thus, it also has more potential for parallelization.
Thus, it enjoys benefits such as higher potential for parallelization.

Local Inconsistency Resolution (LIR) is very expressive.

% Yet often, by placing additional confidence in some parts of the picture,
% or restricting our attention to some small part of it, the inconsistency
% can be calculated more easily.



%% CALL FOR PAPERS WANTS:
% Inference and generating methods for graphs, time series, text, video, and other structured modalities.
% >> will present this in the case study?
%
% Unsupervised representation learning of high dimensional structured data.
% >> a generic algorithm for representation learning:
% >>    describe structure.  Invent or randomize variables + memory architecture.  Choose set of views,
% >>    always including views and
%
% Scaling and accelerating inference and generative models on structured data.
% Uncertainty quantification in AI systems.
%
% Thus, our work may be viewed in several lights.
%     % - a very general probabilistic inference algorithm for structured data;
% On one hand, it is a generic inference algorithm.
%     - a broad class of natural optimizations for inference algorithms;
%     - a generic way of learning


% Many algorithms can be cast in terms of locally minimizing inconsistency.




% \section{Background and Preliminaries}
\section{Background, Preliminaries, and Parametric PDGs}

% $\odot$ denotes pointwise multiplication.

% Each variable $X$ can take on values from a set $\V\!X$ of possible values.
\textbf{Variables and Probability.}
We write $\V\! X$ for the set of values that a (random) variable $X$ can take on,
and $\Delta \V \! X $ for the set of distributions over $\V\!X$.
% Glump glump glump glump glump glump glump.
A conditional probability distribution (cpd) is a map
% (cpd)
$p(Y|X) : \V \!X \to \Delta\mskip-1mu \V \mskip-1mu Y$.
%
% We write $\X$ for 
If $\X = \{X_1, X_2, \ldots\}_{i \in I}$ is an (indexed) set of variables, we regard $\X$ itself as a variable that can take on joint settings $\V\!\X = \prod_{i\in I} \V \! X_i$.
If $\mu \in \Delta \V\!\X$ is a joint distribution and $Y \subseteq \X$, we write $\mu(Y)$ for the marginal of $\mu$ on the variables $Y$. 
% is a distribution over the values of $Y$ for each value of $X$.
%
% \begin{definition}

   A \emph{directed hypergraph}
   % $(N, \mathcal A)$ is a set $N$ of nodes, and a set $\mathcal A$ of \emph{arcs},
   $(N, \mathcal A)$ is a set of nodes $N$ and a set of arcs $\mathcal A$,
   % $(N, \mathcal A)$ is a set of nodes $N$ and a set of arcs $\Ar$,
   % each $a \in \mathcal A$ of which is associated with
   % so that $a \in \Ar$ is associated
   % to each $a \in\! \Ar$ of which we associate
   each $a \in \mathcal A$ of which
   is associated with
   % comes with
   a set $\Src a \subseteq N$ of source nodes,
   and $\Tgt a \subseteq N$ target nodes.
   We also write $\ed {\scriptstyle a}{S}{T} \in \Ar$ to specify an
   arc $a$ together with its sources $S = \Src a$ and targets $T = \Tgt a$.
% \end{definition}

% \paragraph{Geometry.}
\textbf{Geometry.}
% We will need various parameter spaces $\Theta$; to simplify the presentation, assume that parameter spaces are a convex subsets of $\mathbb R^n$ (not necessarily of the same dimension)
% We will need various parameter spaces;
% to simplify the presentation, assume that a parameter space $\Theta$ is a convex subset of $\mathbb R^n$ (not necessarily of the same dimension).
% to simplify the presentation, 
For our purposes, a \emph{pointed parameter space} $\Theta$ is a convex subset of $\mathbb R^n$ for some $n\ge0$
   (that may differ between parameter spaces) with a distinguished default value $\theta_0 \in \Theta$.
   % ; we assume this is the zero vector to simplify notation, but the choice does not matter.
A \emph{vector field} over $\Theta$ is a differentiable
   map $X$ assigning to each $\theta \in \Theta$ a vector $X_\theta \in \mathbb R^n$.
The \emph{gradient} of a twice differentiable map $f : \Theta \to \mathbb R$,
   % written $\nabla_\theta f(\theta)$,
   which we write $\nabla_\Theta f(\Theta)$, is a vector field.
Given a vector field  $X$ and an initial point $\theta_0 \in \Theta$, there is a unique trajectory $y(t)$ that solves the ordinary differential equation (ODE)
$\{\frac{\mathrm d }{\mathrm d t}y(t) = X_{y(t)}$,  $y(0) = \theta_0\}$.
To refer to that solution compactly, we adopt the notatation $\exp_{\theta_0}( X ) := y(1)$.
   % for a compact way of referring to the solution of the ODE.
   % this trajectory.
   % compactly refer to this curve, which is standard in many contexts \cite{lee.smooth-manifolds}.
% At first glance,
% Although on the surface
Although $\exp$ may appear to give us access only to $y(1)$,
   % it is easy to show
    it is easily verified
    that $\exp_{\theta_0}(t X) = y(t)$ for all $t \ge 0$.
% Although it may not be obvious, $\exp$ indeed allows us to describe the entire trajectory, because $\exp_{p_0}(tX) = p(t)$.
Putting all the pieces together: the map $t\mapsto \exp_\theta(t \nabla_\Theta f(\Theta))$ is the smooth path beginning at $\theta$ that follows the gradient of $f$. It is known as
\emph{gradient flow}.

\vfull{
Given a manifold $\Theta$ and a differentiable map $P : \Theta \to \Delta \V\! X$,
    the Fisher Information Matrix
$\mathcal I(\theta)
   % = [ \Ex_{x\sim P(\theta)}[ \frac{\partial^2}{\partial \theta_i \partial \theta_j}  \log p_\theta(x) ]]_{i,j}
$
at each $\theta \in \Theta$
% forms such a Riemannian metric;
gives rise to a Riemannian metric;
thus the mere fact that $\Theta$ parameterizes a family of
probability distributions is enough to make it a Riemannian manifold.
% by $\mat u, \mat v \mapsto \mat u^{\sf T} \mathcal I(\theta) \mat v$.
Moreover, $\mathcal I(\theta)$ is particularly natural in a probabilistic context;
   up to a multiplicative constant, it is the \emph{only} such metric on $\Theta$ that is invariant under sufficient statistics, \cite{chentsov}. \cite{infogoem}
}

% \subsection{PDGs}
% \paragraph{Probabilistic Dependency Graphs.}
\textbf{Probabilistic Dependency Graphs.}
% \paragraph{PDG Syntax.}
A PDG is a
   directed
   (hyper)graph
   whose arcs carry
   probabilistic and causal information, weighted by confidence \cite{pdg-aaai}.
% Equivalent variants of PDGs were defined by \citet{one-true-loss,pdg-infer};
% we give another variant whose explicit parametric nature is useful for our purposes.
We define an unweighted (but for our purposes, equivalent) variant whose explicit parametric nature will prove useful.
% ---%
   % yet it too is essentially equivalent to the others.
% We give yet another variant
% parametric variant of a PDG, which is essentially
   % equivalent (see \cref{appendix:internalization})

% \begin{definition}
%     A PDG $\dg M \!=\! (\X\mskip-2mu, \Ar, \mathbb P, \balpha, \bbeta )$
%     is a directed hypergraph
%         $(\X\mskip-2mu, \Ar)$
%     whose nodes correspond to
%     variables,
%     % together with
%     % probabilities $\mathbb P$
%     %     and
%     %     confidence vectors
%     %     $\balpha \!=\! [\alpha_a]_{a \in \Ar},\bbeta \!=\! [\beta_a]_{a \in \Ar}$,
%     %     so that
%     each arc
%     % $\ed aST \in \Ar$
%     $a \in \Ar$ of which is associated with:
%     % arc $a \in \Ar$ is associated with:
%
%     \begin{itemize}[nosep,itemsep=2pt]
%         \item (subsets of) variables $\Src a, \Tgt a \subset \X$, indicating the respective source and target variables of the edge;
%         % For example,
%         %     $$\Src L = \{A, B\} \ed L{}{} \{C\} = \Tgt L$$
%         %  intuitively represents a joint dependence of $C$ on the variables $A$ and $B$;
%         % \item A manifold $\Theta_a$ of parameters
%         \item
%         a conditional probability distribution
%         {%\subafalse
%             $\p_a(\Tgt a | \Src a)$},
%         on the target variables given
%         the values of the source variables,
%         \item a weight
%         $\beta_a \in \mathbb R \cup \{\infty\}$
%         of confidence in
%         the conditional probability distribution
%         {%\subafalse
%             $\p_a
%             %(\Tgt a | \Src a)
%             $},
%         % \discard{(as measured by the number of independent observations that support $\p_a$), }
%         and
%         \item
%         a weight $\smash{\alpha_a \in \mathbb R}$
%         indicating
%         confidence in the functional dependence of
%         {%\subafalse
%         $\Tgt a\mskip-2mu$ on $\Src a\mskip-2mu$}
%         expressed by $a$.
%         % \discard{
%         % (as measured by the expected number of independent causal mechanisms corresponding to $a$,
%         % that determine $\Tgt a$ given $\Src a$).%
%         % }
%     \end{itemize}
%     % In aggregate, $\balpha = [\alpha_a]_{a \in \Ar}$ and $\bbeta = [\beta_a]_{a \in \Ar}$
%     % are vectors over $\Ar$.
% \end{definition}

% We are particularly interested in PDGs whose cpds we can adjust smoothly.
% In the present paper, we are particularly interested in PDGs whose cpds we can adjust smoothly.
% To that end, we introduce the notion of a \emph{parametric PDG family}
% $\dg P(\Theta)$,
% which is defined the same way, except that each arc $a \in \Ar$ is also associated
% with a manifold $\Theta_a$, and $\p_a$ not itself a cpd, but rather a map from $\Theta_a$ to cpds.
% In other words, we have a cpd
%     % $\p_a^{\theta_a}(\Tgt a | \Src a)$
%     $\p_a(\Tgt a | \Src a; \theta_a)$
%     for each $\theta_a \in \Theta_a$;
%      % ison $\Tgt a$ given $\Src a$,
% thus,
% %
% In addition, we require that there be a special default paramter setting $0_a \in \Theta_a$, such that $\p_a(\Tgt a | \Src a=s; 0_a) \propto \lambda_{\Tgt a}$, for every $s \in \V\Src a$.


\begin{definition}
   % A \emph{parametric PDG family}
   % A \emph{parametric probabilistic dependency graph} (PPDG)
   % A \emph{Parametric Probabilistic Dependency Graph} (PPDG)
   An \emph{unweighted parametric PDG}
   % $\dg P \!=\! (\X\mskip-2mu, \Ar, \mathbb P, \balpha, \bbeta )$
   $
   \dg M(\Theta) =
   (\X\mskip-2mu, \Ar, \Theta= \{\Theta_a \}_{a \in \Ar}, \mathbb P =\{ \p_a \}_{a \in \Ar} )$
   is a directed hypergraph
       $(\X\mskip-2mu, \Ar)$
   whose nodes $\X$ are
   variables,
   and whose arcs
   $a \in \Ar$
   % $\ed aST \in \Ar$
   are each associated with
   % arc $a \in \Ar$ is associated with:
   %
   % \begin{itemize}
           % [nosep,itemsep=2pt,left=0pt]
           % [itemsep=2pt]
           % [itemsep=3pt,parsep=0pt,topsep=0pt]
       % \item (subsets of) variables $\Src a, \Tgt a \subset \X$, indicating the respective source and target variables of the edge;
       % \item 
       a parameter space $\Theta_a$ 
       % with a default value $\theta^{\text{init}}_a$.
       % \item
       % a conditional probability distribution
       % {%\subafalse
       %     $\p_a(\Tgt a | \Src a)$},
       % on the target variables given
       % the values of the source variables,
       % \item
       and
       a map
       $\p_a : \Theta_a \times \V\Src a \to \Delta \V \Tgt a$
       that gives a probability distribution
       $\p_{a}(\Tgt a | \Src a;\theta)$
       over $a$'s target variables given values of its sources and a parameter setting $\theta \in \Theta_a$.

   %     \item a confidence
   %     % $\beta_a \in \mathbb R \cup \{\infty\}$
   %     $\beta_a \in \mathbb R \cup \{\infty\}$
   %     in the cpd
   %     {
   %         %\subafalse
   %         $\p_a(\Tgt a | \Src a)$}%
   %     % \discard{(as measured by the number of independent observations that support $\p_a$), }
   %     % .
   %     ,
   %     and
   % \item
   %     a confidence $\smash{\alpha_a \in \mathbb R}$
   %     in the functional dependence of
   %     {%\subafalse
   %     $\Tgt a\mskip-2mu$ on $\Src a\mskip-2mu$}
   %     expressed by $a$.
   %     % \discard{
   %     % (as measured by the expected number of independent causal mechanisms corresponding to $a$,
   %     % that determine $\Tgt a$ given $\Src a$).%
   %     % }
   % \item a degree of confidence
   %     % $\beta_a \in \mathbb R \cup \{\infty\}$
   %     $\smash{\alpha_a \in \mathbb R}$
   %     in the functional dependence of
   %     {%\subafalse
   %     $\Tgt a\mskip-2mu$ on $\Src a\mskip-2mu$}
   %     expressed by $a$,
   % and a degree of confidence $\beta_a \in [0,\infty]$
   % in the cpd $\p_a$\,.
   %     $\alpha_a$ can be thought of as the number of independent mechanisms by which $\Src a$ determines $\Tgt a$,
   %     while $\beta_a$ can be thought of as the effective number of independent reports endorsing that the probability of $\Tgt a$ given $\Src a$ is $\p_a$.
   % \end{itemize}
   % For completeness, $\Theta = \{\Theta_a \}_{a \in \Ar}$, $\mathbb P = \{ \p_a \}_{a \in \Ar}$.
   % $\balpha = \{ \alpha_a \}_{a \in \Ar}$, $\bbeta = \{\beta_a\}_{a \in \Ar}$.
   %
   % A PDG is the object obtained by fixing the parameters; thus,
   % a
   % joint setting
   % $\theta = (\theta_a)_{a \in \Ar}$ yields a PDG
   % $\dg M = \dg M(\theta)$.
   %
   % We call a PDG \emph{unweighted} if $\balpha=\bbeta=\mat 1$.
   \qedhere
\end{definition}
% \vspace{-0.2ex}

An unweighted PDG is the object obtained by fixing the parameters; thus,
a
joint setting
$\btheta = (\theta_a)_{a \in \Ar}$ yields a PDG
$\dg M = \dg M(\btheta)$.
Thus, at a syntactic level, an unweighted PDG is just an arbitrary collection of cpds. 
When constructing PDGs from known cpds (e.g., $p(Y|X)$), we use the mathematical symbol representing that cpd (e.g., $p$) for both $a$ and $\p_a$. 
We typically depict (parametric) PDGs in graphical notation, specifying a hypergraph 
% drawing
% $f(Y|X)$ is written
% \begin{tikzpicture}[center base]
% 	\node[dpadinline] (X) at (0,0) {$X$};
% 	\node[dpadinline] (Y) at (1.1,0){\small$Y$};
%  	\draw[arr1,->] (X) -- node[fill=white, fill opacity=1, pos=0.35, inner sep=-1pt]{$f$} (Y);
% \end{tikzpicture}\,,
\vspace{-3ex}
\begin{center}
% a cpd
$p(Y|X,Z)$ \;as\;
\begin{tikzpicture}[center base]
   \node[dpadinline] (Y) at (0.95,0) {$Y$};
   \node[dpadinline] (X) at (0,-0.25) {$X$};
   \node[dpadinline] (Z) at (0, 0.25) {$Z$};
   \coordinate (ctr) at (0.38,0);
   \cmergearr[arr2] {X.0}{Z.0}Y{ctr}
   \node[above right=-1pt and -3pt of ctr] {$p$};
\end{tikzpicture}\,
% ~~\text{and $q(Y)$ as }~~
% \begin{tikzpicture}[center base]
% 	\node[dpad0] (Y) at (1,0) {$Y$};
% 	\draw[arr2, <-] (Y) -- node[left,inner sep=2pt]{$q$} ++(0,1);
% \end{tikzpicture}~.
% ~\text{and $q(A,B)$ as }~
$\quad$and$\quad$ $q(A,B)$ \;as\;
\begin{tikzpicture}[center base]
   \node[dpadinline] (A) at (1,0.25) {$A$};
   \node[dpadinline] (B) at (1,-0.25) {$B$};
   \coordinate (tip) at (0.15, 0);
   \coordinate (center) at (0.45,0);
   % \draw[arr2, <-] (Y) -- node[left,inner sep=2pt]{$q$} ++(0,1);
   \cunmergearr[arr1] {tip}{A.180}{B.180}{center}
   \node[above left=0pt and 0pt of center,inner sep=2pt]{$q$};
\end{tikzpicture}\,.
\end{center}
%
% By default, take $\beta,\alpha{=}1$.
% Unless otherwise specified, take $\balpha = \bbeta = \mat 1$ by default---although the choice of $\balpha$ will matter only in \cref{sec:factor-graph}.
   % , but we will sometimes specify ``high confidence''
   % ($\beta{=}\infty$).
\vfull{
Given PDGs $\dg M_1$ and \vphantom{$\big|$}$\dg M_2$,
   let denote $\dg M_1 + \dg M_2$ the PDG with their combined information, e.g., by taking the disjoint union of their arcs. 
}
% , or reweighting of a PDG's confidences.
% % If $\psi \in \mathbb R^{2\Ar}$,
% % If $\psi \in \mathbb R^{\Ar+ \Ar}$,
% If $\psi \in \mathbb R^{\Ar}$,
% we write $\psi \odot \dg M$ for the PDG $\dg M'$ re-weights $\bbeta$,
%     multiplying by $\psi$ pointwise.
%     %  i.e., $\beta_a' = \beta_a \psi_{0,a}$ and $\alpha_a' = \alpha_a\psi_{1,a}$.
% We the same symbol for pointwise vector multiplication generally.
% % \begin{itemize}[nosep,itemsep=2pt]
% %     \item a space $\Theta_a$ of parameters, and a
% % \end{itemize}

It is not hard to see that a PDG can be faithfully viewed as the special case of a parametric PDG in which every 
$\Theta_a = \{ \theta_0 \}$ is a singleton containing only the default value.
Conversely, a parametric PDG may be viewed as a PDG by adding each $\Theta_a$ as an explicit variable.


% \paragraph{Semantics.}
% What makes PDGs interesting are
\textbf{PDG Semantics and Inconsistency.}
The power of PDGs comes from their semantics, which sew their (possibly inconsistent)
cpds and confidences together into joint probabilistic information.

A PDG contains two kinds of information: qualitative information about
the types of causal mechanisms (the graph $\Ar$),
and observational data (the cpds $\mathbb P$).
In the standard presentation, a PDG also comes with weight vectors  $\balpha,\bbeta \in \smash{\overline{\mathbb R}}^\Ar$ over the arcs, encoding confidence in information of each type: $\alpha_a$ can be thought of as the number of independent causal mechanisms by which $\Src a$ determines $\Tgt a$, while $\beta_a$ can be thought of as the effective number of independent reports endorsing that the probability of $\Tgt a$ given $\Src a$ is $\p_a$.
Corresponding to these two types of information, PDG semantics provide a way of scoring compatibility of a joint distribution $\mu(\X)$ with information of each type.

% Corespondingly, we can score a candidate joint distribution $\mu \in \Delta \V\!\X$ by its incompatibility with each kind of information.
% This is done by scoring joint distributions $\mu \in \Delta \V\!\X$
   % by their incompatibility
With respect to a PDG $\dg M$,
the \emph{observational incompatibility}
of a joint probability measure
$\mu \in \Delta \V\!\X$ is given by
a weighted sum of relative entropies
\begin{equation}
\OInc_{\dg M}(\mu) :=
   % \sum_{L \in \Ed} \beta\ssub L\, \kldiv[\Big]{\mu(\Tgt L,\Src L)}{p\ssub L(\Tgt L | \Src L) \mu(\Src L)}.
   %oli5:
   % \sum_{a \in \Ar}
   \!
   \sum_{\smash{\ed aST \mathrlap{\,\in \Ar}}} \subafalse
   \beta_a\, \kldiv[\Big]{\mu(\Tgt a,\Src a)}{\p_a(\Tgt a | \Src a) \mu(\Src a)}
       ,
       \label{eq:oinc}
\end{equation}
and can be thought of as the excess cost of using codes
optimized for our beliefs $\mathbb P$ weighted by the confidence we have in them, when in fact $\X \sim \mu$.
If $\bbeta > \mat 0$, then $\OInc_{\dg M}(\mu) = 0$ if and only if 
$\mu$ satisfies the constraints imposed by every cpd of $\mathbb P$.
% $\mu(Y|X) = \p_a$ for all 
Intuitively, the cpds of $\mathbb P$ are inconsistent when constraints are not simultaneously satisfiable,
in which case the \emph{observational inconsistency} $\aar{\dg M}_0 := \inf_\mu \OInc_{\dg M}(\mu)$ 
turns out to be a particularly important measure the magnitude of the unavoidable internal conflict between the probabilistic data in $\dg M$.
% turns out to be a particularly useful measure of the magnitude of the gap. 
%
%
Modern machine learning has a tendency to prize observational data above all else, and in particular above structural informaion such as causal influence or qualitative independencies; for this reason, the scoring function $\OInc$ and the corresponding observational inconsistency $\aar{\dg M}_0$ will suffice for most of our examples. 
% Moreover, it is possible 

Nevertheless,
we can also score $\mu$ by its incompatibility
with the structural information in the weighted hypergraph $(\Ar,\balpha)$.
This \emph{structural deficiency} is given by:%
   \footnote{If the underlying variables are continuous, then we redefine entropy
   as follows. Assume each variable comes with a base measure $\lambda$, like the Lebesgue or counting measure.
   Then define $\H_\mu(X) := \Ex_\mu[ \log \frac{\mathrm d\mu(X)}{\mathrm d \lambda_X}]$, where $\frac{\mathrm d\mu}{\mathrm d\lambda}$ is the Radon-Nikodym derivative of $\mu(X)$ with respect to $\lambda_X$.}
   % with respect to a PDG $\dg M$ is
\begin{align*}
   % \SInc_{\dg M}(\mu) := - \H(\mu) + \sum_{L \in \Ed} \alpha\ssub L\, \H_\mu(\Tgt L | \Src L),
   \SDef_{\!\Ar,\balpha}(\mu) &:=
   % \SDef_{\!(\Ar,\balpha)}(\mu) &:=
       % - \H(\mu) + \sum_{a \in \Ar} \alpha_a\, \H_\mu(\Tgt a | \Src a).
       % \pqty[\Big]{\; \sum_{\ed aST \mathrlap{\,\in \Ar}}\subafalse \alpha_a\, \H_\mu(\Tgt a | \Src a) } - \H(\mu).
       %%%%
       % \kldiv{\mu}{\lambda_{\X}} -
       % \sum_{\ed aST \mathrlap{\,\in \Ar}}\subafalse \alpha_a\,
       %     \kldiv{\mu (\Tgt a\Src a)} { \lambda_{\Tgt a} \mu(\Src a) }
       \numberthis
        \label{eq:sdef}
       %%%%
       % \\ &=
   %         \Ex_{\mu}  \bigg[ \log\; \frac{\mu(\X)}{\lambda(\X)}
   %         % \prod_{\vphantom{\big|}\ed aST \mathrlap{\,\in \Ar}}\subafalse
   %         \prod_{\ed aST }\subafalse
   %             \left(\frac{\lambda(\Tgt a|\Src a)}{\mu(\Tgt a | \Src a)}\right)^{\!\alpha_a}
   %         \bigg]
   % ,
       - \H(\X) + \sum_{a \in \Ar} \alpha_a \H(\Tgt a \mid \Src a)
       % \Ex_{\mu}  \bigg[ \log\; \frac{\mu(\X)}{\lambda(\X)}
       % % \prod_{\vphantom{\big|}\ed aST \mathrlap{\,\in \Ar}}\subafalse
       % \prod_{\ed aST }\subafalse
       %     \left(\frac{\lambda(\Tgt a|\Src a)}{\mu(\Tgt a | \Src a)}\right)^{\!\alpha_a}
       % \bigg]
   ,
   % \vspace{-1ex}
\end{align*}
% can be thought of as measuring how effectively $\mu$ factors along the arcs $\Ar$.
and, roughly speaking, measures $\mu$'s failure to arise as a result of
   an independent causal mechanism along each hyperarc \cite{QIM}.
% There is not space to motivate it fully, but here are a few nice properties.
If $\Ar$ is a qualitative Bayesian Network and $\balpha = \mat 1$, for instance,
   then $\SDef_{\!\Ar,\balpha}(\mu) \ge 0$ with equality
   iff $\mu$ has the conditional independencies of $\Ar$.
% It turns out that this information can also be expressed
% We point the reader to
%  \citet{pdg-aaai,one-true-loss,pdg-infer}
% % for further details, examples, and motivation.
% for more details, motivation, and examples.
% We encourage the reader to consult previous work for further details.

% that if $\Ar$ has the structure of $\SDef_{\!\dg M}(\mu) > 0$
% which itself has a confidence $\gamma$,


% \begin{align*}
%     \bbr{\dg M}_\gamma&(\mu)
%         := \OInc_{\dg M}(\mu) + \gamma \, \SInc_{\dg M}(\mu)
%             \numberthis\label{eqn:scoring-fn} \\[-0.2ex]
%         % =& \Ex_{\mu}\left[\, \sum_{L \in \Ed} \log \frac
%         %     {\mu(\Tgt L| \Src L)^{\beta\ssub L - \gamma \alpha \ssub L}}
%         %     {p\ssub L(\Tgt L | \Src L)^{\beta \ssub L}}
%         % \right] - \gamma \H(\mu)
%         &= \Ex\nolimits_{\mu}\bigg[
%             \,
%             % \gamma \log \mu(\X) +
%             % \sum_{a \in \Ar}
%             \sum_{\ed aST \mathrlap{\, \in \Ar}} \subafalse
%             % \sum_{a,S,T,\alpha,\beta,p \in \hat{\dg M}} \subafalse
%                 % \let\plainbeta\beta \def\beta_a{\plainbeta} \let\plainalpha\alpha\def\alpha_a{\plainalpha} \def\p_a{p}
%             \log \frac
%             {\mu(\Tgt a| \Src a)^{\beta_a - \gamma \alpha_a}}
%             {\p_a(\Tgt a | \Src a)^{\beta_a}}
%         \bigg] - \gamma \H(\mu)
% \end{align*}

With confidence $\gamma \ge 0$ in the structural information overall,
the $\gamma$-\emph{inconsistency} of $\dg M$ is the smallest possible overall incompatibility of any distribution with $\dg M$, and denoted
% \vskip-1ex
\begin{equation}\label{eq:inconsistency}
   \aar[\big]{\dg M}_\gamma := {\inf_{\mu}\,  \Big( 
       % \OInc_{(\mathbb P,\bbeta)}(\mu)
       \OInc_{\dg M}(\mu)
        +  \gamma \, \SDef_{\!(\Ar,\balpha)}(\mu) \Big).}
\end{equation}
\citet{one-true-loss} argues that this inconsistency measure
   % $\aar{- }_\gamma$
   \eqref{eq:inconsistency}
   is a ``universal'' loss function, largely by showing how it
   specializes to standard loss functions across a wide breadth of contexts.
It follows that, at least at an abstract level,
   much of machine learning can be viewed as inconsistency resolution.
%
% We take this idea one step further, operationalizing the resolution process.
We take this idea a few steps further, by operationalizing the
   resolution process,
   % and making it even more expressive with an approximate, local variant
   and allowing it to be done ``locally''.
% The latter dramatically expands what computations can be expressed.
   % significantly  expanding what can be expressed.
%
% We operationalize resolution process, and then extend this idea to allow for locality.
   % how even more can be done if we allow
%
   % can be seen as \emph{local} resolution.
% We then demonstrate that several very different historically important
   % procedures are special cases.
    % of our algorithm.
   % of an algorithm we call \emph{local inconsistency resolution}.
But what exactly do we mean by that?

\vfull{
PDG inference is fixed-parameter tractable:
assuming a graph with bounded tree-width, the  can be computed in polynomial time.
and it appears that \cite{pdg-infer}.
Several important algorithms can alredy be seen as inconsistency reduction.
When viewed as a graphical model}

% A surprisingly broad class of algorithms already
% \begin{itemize}
%     [nosep]
%     \item Belief updating (Conditioning, Jeffrey's rule)
% \end{itemize}
% However, global inference




\section{The Local Inconsistency Resolution (LIR) Algorithm}
% Two knobs to control the model. The first one controls what you're looking at and what you get to change.
% We now describe the local inconsistency reduction algorithm.
% There are two distinct senses in which inconsistency resolution can be done locally:
%     we can either restrict how much we can see, or how much we can do.
% There are two distinct senses in which inconsistency can be resolved locally:
%     we can restrict what we can see, or what we can do.
% There are two aspects of resolving inconsistency: what
% We can either
There are two distinct senses in which inconsistency resolution can
   be \emph{local}: we can restrict what we can see, or what we can do about it.
Correspondingly, there are two ``focus'' knobs for our algorithm:
   one restricts our \textbf{attention} to the inconsistency of a subset of arcs $A \subseteq \Ar$,
   % and one that restricts our control to the parameters of a subset
   and the other restricts our \textbf{control} to only the parameters of a subset
   $C
   % (\subseteq A)
    % \subseteq \Ar$
    \subseteq \Ar$
     as we resolve that inconsistency.
% \begin{enumerate}[label=(\alph*)]
% (a)
% \item
   % restrict our attention to a subset $A \subseteq \Ar$ of arcs,
   % restrict our attention to the inconsistency of a subset of arcs $A \subseteq \Ar$,
   %     % trying to reduce inconsistency of the sub-PDG,
   %     or
   %     % taking a the inconsistency of a
   %     \label{item:attn-restrict}
   % % (b)
   % % \item
   %     restrict our control to the parameters
   %     % $\Theta_C := \prod_{a \in C}\Theta_a$ of a subset $C \subseteq \Ar$ of arcs.
   %     of arcs $C \subseteq \Ar$.
   %     \label{item:ctrl-restrict}
% \end{enumerate}
% In the first case,
% Both make the problem more tractible.
% Restricting attention \ref{item:attn-restrict}
% In other words,
The former makes for an underestimate of the inconsistency that is easier to calculate, while
% restricting control \ref{item:ctrl-restrict}
the latter
makes for an easier optimization problem.
These restrictions are not just cheap approximations, though:
   they are also appropriate modeling assumptions for
   actors that cannot see and control everything at once.

% \subsection{Attention and Control Masks}

Attention and control need not be black and white.
% More general than selecting $A,C \subseteq \Ar$,
% A generalization of $A,C \subseteq \Ar$ is
A more general approach would be to choose
   % an attention mask $\varphi \in \mathbb R^{\Ar}$ and
   % a control mask $\chi \in [0,\infty]^{\Ar}$.
   an \emph{attention mask} $\varphi \in \mathbb R^{\Ar}$ and
   a \emph{control mask} $\chi \in [0,\infty]^{\Ar}$.
Large $\varphi(a)$ makes $a$ salient while $\varphi(a) \!=\! 0$ keeps it out of mind;
similarly, large $\chi(a)$ gives significant freedom to change $a$'s parameters,
small $\chi(a)$ affords only minor adjustments, and $\chi(a) \!=\! 0$
   % means we cannot change them at all.
   prevents change altogether.
% $\gamma$ may be viewed $\varphi$
% So, $\chi$ controls locality in the sense of ``local search''.
%
% So overall, the algorithm goes like this.
   % We recieve input in the form of a PDG $\Ctx$, and
   % a memory layout $\dg M(\Theta)$.
% We then repeatedly select an attention mask $\varphi$ and a control mask $\chi$, and update
% Once we select attention and control masks,
% % We then repeatedly select attention and control masks, and update
%     we update the parameters we can control so as to reduce the inconsistency of
%         what is in view.
% In more detail:

In full generality, we refine the types of $\varphi$ and $\chi$ one step further by breaking the control and attention in each arcs into natural subcomponents. 
The behavior of the attention mask $\varphi$ described in the previous paragraph matches the role played by $\balpha$ and $\bbeta$ in PDG semantics---which why we needed only to define unweighted PDGs as our basic objects, but require weighted semantics for them.
The parameter $\gamma$ also describes a form of attention, to the overall qualitative information in the network. Thus we take our control mask $\varphi = (\balpha,\bbeta,\gamma)$ to be a vector of dimension $2|\Ar|+1$, 
and write $\aar{\varphi \odot \dg M} := \aar{\dg M,\balpha,\bbeta}_\gamma$ to denote the $\gamma$-inconsistency of the weighted PDG $(\dg M, \balpha,\bbeta)$. 
Because we will typically set $\gamma=0$, rendering $\balpha$ irrelevant (except in \cref{sec:factor-graph} where $\gamma=1$ and $\balpha=\bbeta$), we assume that $\balpha = \bbeta$ unless otherwise specified. 
% and write $\varphi(a) = 1$ for $\alpha_a = \beta_a = 1$. 
We also allow $\chi(a)$ to be a vector of dimension $\dim \Theta_a$, so overall $\chi$ is a vector of dimension $\sum_{a \in \Ar} \dim \Theta_a$.
%
% A few of our most bizarre examples require that $\varphi$ and $\chi$ to depend on $\theta$.


% At a high level, the algorithm proceeds as follows.
% The full algorithm, formalized in \cref{algo:LIR},
The full procedure,
   formalized in \cref{algo:LIR},
   is a heuristic algorithm that adjusts the parameters of a parametric PDG $\MThetadense$
   as to locally resolve inconsistencies as follows.
First, 
% receive context PDG $\Ctx$ and 
% initialize mutable memory $\MThetadense$.
accept a belief state in the form of a parametric PDG $\dg M(\Theta)$ and initialize the parameters to their default values.
   % a PPDG that whose parameters we will be changing to reduce its inconsistency with $\Ctx$.
%In each iteration, choose $\gamma$, a control mask $\chi$ over $\MThetadense$, and an attention mask $\varphi$ over $\MThetadense$ and $\Ctx$.
In each iteration, choose \emph{focus} consisting of an attention mask $\varphi = (\balpha,\bbeta,\gamma)$ and a control mask $\chi$.
%
% Use methods of \citet{pdg-infer} to calculate
Calculate
% the inconsistency of $\varphi\odot(\dg M(\Theta) + \Ctx)$,
$\aar{\varphi\odot\MThetadense}_\gamma$, the inconsistency of
   the combined context and memory, weighted by attention.
In many cases (e.g., when $\bbeta \ge \gamma\balpha$) these problems can be solved with techniques in conic optimization \citep{pdg-infer},
   but this may be intractable unless the attention is narrow enough or one can find a formula for it in closed form.
Finally, mitigate this local inconsistency
   by updating mutable memory via (an approximation to) gradient flow,
   changing the parameters associated with $a$ in proportion to the degree of control $\chi(a)$.
       % context and memory $\Ctx + \dg M(\theta)$
% This procedure is fully formalized in \cref{algo:LIR}.
% This is formalized in \cref{algo:LIR}.

% To describe the algorithm formally in full generality, we also need to des
% If $\dg M(\Theta)$ is a parametric PDG family, then we write
% $\dg M.\Ar$
% $\Ar(\dg M)$
% for the set of arcs in the underlying PDG, and
% $\dg M.\boldsymbol\theta$

\begin{algorithm}
   \caption{Local Inconsistency Resolution (LIR)}
   \label{algo:LIR}
   \begin{algorithmic}
       % \STATE \textbf{Input:}
       %     mutable memory $\dg M(\Theta) $,
       %     $\gamma \ge 0$,
       % \STATE ~~~~ procedure $\textsc{Refocus}: () \to PDG \times [0, \infty]^{\Ar(\dg M)}$
       % \STATE \textbf{Input:}
       %     context $\Ctx : $ PDG
       % \STATE ~~~~
       %     mutable memory $\Mm(\Theta) : $ PPDG
       \STATE \textbf{Require:} procedure {\sc Refocus}
       \STATE \textbf{Input:}
           % context $\Ctx$,~ 
           % mutable memory $\MThetadense$.
           knowledge base $\MThetadense$, 

       % \STATE \textbf{Variables:}
           % state $\theta \in \Theta$,
           % PDG $\Ctx$.
       % \smallskip

       \STATE Initialize $\btheta^{(0)} \gets 
           % \theta^{\text{init}}$;
           0$;
       % \STATE Initialze $\theta \gets 0$;

       \FOR{$t = 0, 1, 2, \ldots$}
           % \STATE $\Ctx \gets \textsc{Refresh}(\Ctx)$;
           %     $\qquad${\color{gray}\small\texttt{//optional}}
           \STATE $\varphi, \chi
               \gets \textsc{Refocus}()$;
               % \hfill{\color{gray}\small\texttt{//get loci of focus, control}}
           \STATE $\btheta^{(t+1)} \!\gets \exp_{\btheta^{(\mskip-1mut\mskip-1mu)}}\!
               % \Big\{\! {-} \chi \odot \nabla_{\!\Theta}
               \Big\{\! {-} \chi \odot \nabla_{\!\btheta}
               % \aar[\Big]{ \varphi\odot \MThetadense } \Big\}$;
               \aar[\Big]{ \varphi\odot \dg M(\btheta) } \Big\}$;
               % \aar[\big]{ \Ctx + \varphi\odot \dg M(\theta) }_{\!\gamma} \Big\}$;
       \ENDFOR
   \end{algorithmic}
   % \small\color{gray}
   % Issues. Choice of $\Ctx$ includes choice of $\varphi$, if $\varphi : \Ar(\dg M + \Ctx) \to \mathbb R$.
   % But if $\varphi : \Ar(\dg M) \to \mathbb R$, they're not overlapping, and $\chi,\varphi: \mathbb R^{\Ar(\dg M)}$ look more like ``duals''.
   % Of course, we can't then eliminate the refresh instruction.  I also don't like the interpretation as much: if your focus changes in a static environment, I want that to be reflected in $\varphi$, not changing context.
\end{algorithm}
Before we can run \cref{algo:LIR}
we 
% we must supply two more procedures:
% \textsc{Refresh}, to get new context in online settings (the identity by default), and
\textsc{Refocus}, to select attention and control masks.
% Different choices result in different algorithms;
   % for our purposes,
We focus on the case where \textsc{Refocus}
   chooses non-deterministically
   from a fixed finite set of attention-control mask pairs $\mathbf{F}
   = \{ (\varphi_i,\chi_i) \}_{i=1}^n
$, which we call \emph{foci}.
% The other procedure, \textsc{Refresh} is the identity by defualt, and can be removed in theory by making supplying an infinitely large initial context containing every possible result of refresh.
% The other procedure,
% \---that is, procedures that continually recieve inputs.

The ODE described in the final line may be approximated with an inner loop running an iterative
   gradient-based optimization algorithm.
Alternatively, if \textsc{Refocus} produces small $\chi$,
   then it is well-approximated by a single gradient descent step of size $\chi$.
At the other extreme, if $\chi$ is infinite in every component, then
   % so long as the parameterizations $\mathbb P$ are log-concave,
% then $\chi=\infty$ finds
   % a parameter setting that globally minimizes inconsistency
we typically expect
the final line to reduce to
   \begin{equation}
       \btheta^{(t+1)} \gets \arg\min_{\btheta} \,
           \aar[\big]{ \varphi\odot \dg M(\btheta) }\,,
           % ~~\text{because of}
           \label{eqn:solve-opt}
   \end{equation}
at least in many cases of interest. 
For example:

\begin{linked}{proposition}{logccave}
   % If $\mathbb P$ is log-concave, then
   If $\dg M(\Theta)$ is an unweighted parametric PDG whose parameterizations
   $\mathbb P$ are either constant or unconditional and log-concave and
   % $\varphi = (\balpha,\bbeta,\gamma) \ge 0$, then for small enough $\gamma$, 
   $\bbeta \ge \gamma\balpha$,
   the map $\btheta \mapsto \aar{\dg M(\btheta),\balpha,\bbeta}_\gamma$ is convex.%
   \onlyfirsttime{\footnote{All proofs can be found in the appendix.}}
   % and strictly so for $\gamma > 0$.
\end{linked}
% \begin{proof}
%     It is known that $p \mapsto \aar{\dg M + p}$ is convex (see appendix),
%     and also that $\aar{\dg M + p}$
%     % Let $\varphi_C$ be components of $\varphi$ that act on the arcs of $\dg M$.
% \end{proof}
%
%
% We now consider some important speical cases.
To our surprise (and contradicting a prior version of this paper), we have found that inconsistency is not always convex in the parameters of \emph{conditional} probability distributions---yet we conjecture that \eqref{eqn:solve-opt} holds nevertheless. Determining whether or not $\aar{\dg M + p(Y|X)}$ is quasi-convex in $p$ remains a key open question in the theory of PDGs.
%
In the remaining sections, we give a sample of
some historically important algorithms that are instances of LIR.

% We now consider some special cases of \cref{algo:LIR}.

%%%%%
% If $\Ctx$ is a PDG with variables $\X$,
% and our memory $\dg M$ contains a single joint distribution $\mu(\X)$
% then $\textsc{LIR}(\Ctx, \dg M, \gamma)$ equals $\bbr{\Ctx}^*_\gamma$, the optimal distribution
%%%%%
% If $\Ctx$ is a PDG with discrete variables $\X$,
% $\dg M(\Theta)$ consists of a single joint distribution $\mu(\X)$
%     parameterized as a vector $[0,1]^{\V\!\X}$,
% and \textsc{Refocus} always produces constant $\chi,\gamma$,
% then $\textsc{LIR}(\Ctx, \dg M)$ equals $\bbr{\chi\odot\Ctx}^*_\gamma$,
% the optimal distribution, albeit in a roundabout manner.
%%%%%
\vfull{
   If $\dg M$ is a PDG with discrete variables $\X$
   and we regard $\mu(X)$
   $\dg M(\Theta)$ consists of a single joint distribution $\mu(\X)$
       parameterized as a vector $[0,1]^{\V\!\X}$,
   and \textsc{Refocus} always produces constant $\chi,\gamma$,
   then $\textsc{LIR}(\Ctx, \dg M)$ equals $\bbr{\chi\odot\Ctx}^*_\gamma$,
   the optimal distribution, albeit in a roundabout manner.
   }
% in which (i.e., every $A_t = \Ar$) amounts to \emph{global} inconsistency reduction, i.e.,
% the universal training objective of \cite{one-true-loss}. which is \#P-hard.
% In this case, supposing $\bbeta \ge 0$,
% then the global inconsitency $\aar{\dg M}
% $ can only decrease in time
% It is worth noting that LIR does not always converge.

% \section{Warmup: Control in a Classification Setting}
\section{LIR in the Classification Setting}
Consider a parametric classifier $p_\theta(Y|X)$, perhaps
   arising from a neural network whose final layer is a softmax,
   and suppose we also have a labeled example $(x,y)$. 
Let's capture this situation with a parametric PDG.
Since $\V Y$ is a discrete label space space, we regard labels as unconditional distributions over $Y$, viewing ``hard'' labels such as our $y \in \V Y$ as vertices of this simplex, while also allowing for label smoothing \citep[see][for an overview]{muller2019does}.
%
Doing the same for $x$ may be intractable, but fortunately is not necessary; in the typical case where the space of inputs $\V\!X$ is itself a manifold (e.g., color images in $[0,1]^{W\times H\times 3}$), we can regard a value $x \in \V\!X$ as parameterizing a deterministic unconditional probability over $X$.
To communicate this deterministic parameterization visually, we use a double-headed arrow, writing
$\smash{
\tikz[center base]{\node[dpadinline](X) {$X$}; \draw[arr1, <<-](X) to node[fill=white,inner sep=1pt,pos=0.65]{$x$} +(-1.1,0);}
}$.
% Viewing $\Theta$ as a variable, the discrimainator
% as a conditional probability $p(Y| X, \Theta)$.
% Together with a labeled sample $(x,y)$,
%  and parameter setting $\theta$,
Combining the parametric classifier $p_\theta(Y|X)$ with the sample $(x,y)$,
   we get a parametric PDG
%
% \[ %%%>>  EXPLICIT PARAMTERIZATION; THETA IS A VARIABLE  <<%%
%     \dg M := ~~
%     \begin{tikzpicture}[center base]
%         \node[dpad1] (X) at (0,0) {$X$};
%         \node[dpad1] (Y) at (2,0) {$Y$};
%         \node[dpad1] (T) at (1, 0.7){$\Theta$};
%         \cmergearr[arr1]XTY{1.2,0}
%         \draw[arr1, <<-] (T) to node[above,pos=0.65]{$\theta$} +(-1,0);
%         \draw[arr1, <<-] (X) to node[above,pos=0.65]{$x$} +(-1,0);
%         \draw[arr1, <-] (Y) to node[above,pos=0.65]{$y$} +(1,0);
%         \node[anchor=south,inner sep=2pt] at (0.8,0) {$p$};
%     \end{tikzpicture},
% \]
\begin{equation}
% $
   \dg M(x,y,\theta) := \;
   \begin{tikzpicture}[baseline=-0.6ex]
       % \node[dpad1] (X) at (0,0) {$X$};
       % \node[dpad1] (Y) at (1.5,0) {$Y$};
       % \draw[arr1, <<-] (X) to node[above,pos=0.65]{$x$} +(-1,0);
       % \draw[arr1, <-] (Y) to node[above,pos=0.65]{$y$} +(1,0);
       \node[dpadinline] (X) at (0,0) {$X$};
       \node[dpadinline] (Y) at (1.1,0) {$Y$};
       \draw[arr1, <<-] (X) to node[above,pos=0.65]{$x$} +(-0.85,0);
       \draw[arr1, <-] (Y) to node[above,pos=0.65,inner sep=2pt]{$y$} +(0.75,0);
       \draw[arr1] (X) to
           node[above] {$p_\theta$} (Y);
   \end{tikzpicture}
   \quad\cong\quad
       \begin{tikzpicture}[center base]
       \def\spray{.24}
       \node[dpad0] (X) at (0,-0.4) {$X$};
       \node[dpad0] (Y) at (1.2,0) {$Y$};
       \node[dpad0] (T) at (0, 0.4){$\Theta$};
       % \draw[arr1,-,shorten >=3pt,transform canvas={yshift=-1pt}, defv,dashed,ultra thick] (X) to (Y);
       % \draw[arr1,-,shorten >=3pt,transform canvas={yshift=0.75pt}, atkv] (X) to (Y);
       % \draw[arr1] (X) to node[above] {$p$} (Y);
       \mergearr[arr1,gray]XTY
       \node[above right=0pt and 0pt of center-XTY]{$p$};
       \draw[arr1, <<-] (X) to node[above,pos=0.65]{$x$} +(-0.85,0);
       \draw[arr1, <<-] (T) to node[above,pos=0.65]{$\theta$} +(-0.85,0);
       \draw[arr1, <-] (Y) to node[above,pos=0.65,inner sep=2pt]{$y$} +(0.75,0);
   \end{tikzpicture}
   \label{pdg:classfication}
% $
\end{equation}
whose observational inconsistency is
$
   % \aar{\dg M(x,y,\theta)}_0 = 
   - \log p_\theta(y|x)
$, the standard training objective for such a classifier \cite{one-true-loss}.
% This inconsistency is irreducible, since removing any of
% If $X$ and $\Theta$ are expressive enough, any of the four edges will make the inconsistency disappear
% All four components play a major role in giving rise to this inconsistency.
Each cpd plays major role in this inconsistency.
% Fixing all of $\dg M$ as our context,
% Fixing uniform global attention $\varphi = 1$,
What happens when we resolve this it
with control over different arcs?

\begin{itemize}[nosep,itemsep=3pt, left=2em]
   \item Adjusting $\theta$
   % (i.e, $C = \{\theta\}$)
   amounts to \textbf{training} the network in the standard way.
       % In more detail, $\chi$
       % In this case, the value $\chi$ of the control mask corresponds to the number of optimization iterations.
       In this case, the value $\chi$ of the control mask corresponds roughly
       to the product of the learning rate and the number of optimization iterations spent on the example $(x,y)$.

   \item Adjusting $y$
   % (i.e., $C = \{y\}$)
   amounts to \textbf{inference}:
   % is like a  forward pass, in that it adjusts $y$ to match distribution $p_\theta(Y|x)$.
   it adjusts $y$ to match distribution $p_\theta(Y|x)$.
       % In particular, $\chi=\infty$ corresponds to selection $y(
       % If there is also a qualitative edge that states that $Y$ is deterministic

   \item Adjusting $x$
   % (i.e. $C = \{x\}$ )
   amounts to \textbf{forming an adversarial example}: it makes small changes to the input
   until the (fixed) network gives it the desired label.
   % amounts to constructing an adversarial example---that is, fixing the network
   % If $X$ represents images, for example, this means fixing the network and making small adjustments to the image until the network assigns it the desired label.
\end{itemize}

Some readers may find the last point surprising. 
While adversarial examples \citep{goodfellow2014explaining} are often presented as weird quirk of neural networks,
   they are, together with training and inference, one of the three basic resolutions ways of resolving this simple inconsistency.
% 
The sustained interest of the machine learning community on adversarial examples
   may appear to be a cultural phenomenon,
   % but in some sense it is .
   % but at a mathematical level, it is no accident.
   but at a mathematical level, it is no accident \citep{shafahi2018adversarial}.
At this level of abstraction, there is no difference between
   the network parameters and inputs.
% We can highlight this visually by observing that, after adding L2 regularization \cite{one-true-loss}, translating the previous diagram to an ordinary PDG gives us the symmetrical:
Making the parameterization of $p$ explicit and adding L2 regularization,
the symmetry
% between parameters and inputs
becomes striking (see \cref{fig:adv}, right).



With minor modifications to $\dg M(x,y,\theta)$, we can capture standard procedures that are more common in the training process.

% \textbf{classification}
% Furthermore, these .
% with memory $\dg M(\Theta) = p$ and context
%  $y_i$ for every possible
% \textbf{adversarial attack in practice.}
% Take mutable memory to also include a special arc $\ed {\mathit{adex}}{}X$,
% and that is the focus
% \textbf{A more realistic classification setting.}
\textbf{Stochastic Gradient Descent.} \label{sec:SGD}
Take the mutable state to be the classifier $p$ as before.
Define $\textsc{Refresh}$ so that it draws a batch of samples $\{(x_i,y_i)\}_{i=1}^m$,
and returns a PDG with a single arc describing their emperical distribution $d(X,Y)$, and
% Define $\textsc{Refocus}$ so that it sets $\varphi = \{d \mapsto \infty, p \mapsto 1\}$
%     (corresponding to high confidence in the data, default confidence in the still-training classifier)
let $\textsc{Refocus}$ be such that $\varphi(d) = \infty$
   (reflecting high confidence in the data).
If $\eta := \chi(p) \varphi(p)$ is small, then
   LIR reduces to stochastic gradient descent with batch size $m$ and learning rate $\eta$.


% \textbf{Adversarial training views.}
% \textbf{SGD: batches as context.}
\begin{figure}
   \centering
   % \tikzset{atkv/.style={red!50!white},defv/.style={green!50!gray}}
   \tikzset{atkv/.style={green!70!black},defv/.style={blue}}
   \begin{tikzpicture}[center base]
       % \node[dpad1] (X) at (0,1) {$X$};
       % \node[dpad1] (X') at (0,0) {$X'$};
       \node[dpad1] (X) at (0,0) {$X$};
       \node[dpad1] (Y) at (1.3,0) {$Y$};

       \draw[arr1,-,shorten >=3pt,transform canvas={yshift=-1pt}, defv,dashed,ultra thick] (X) to (Y);
       \draw[arr1,-,shorten >=3pt,transform canvas={yshift=0.75pt}, atkv] (X) to (Y);
       \draw[arr1] (X) to node[above] {$p$} (Y);
       % \draw[arr1] (X) to node[right] {$\mathcal N$} (X');
       % \node[dpad1] (T) at (1, 0.7){$\Theta$};
       % \cmergearr[arr1]XTY{1.2,0}
       % \draw[arr1, <<-] (T) to node[above,pos=0.65]{$\theta$} +(-1,0);
       \coordinate (xend) at ($(X)+(-1.1,0.5)$);
       \draw[arr1,-,shorten <=5pt,transform canvas={yshift=0.75pt}, atkv] (X) to (xend);
       \draw[arr1, <-] (X) to
           %  node[left,pos=0.85]{$\mathcal N(x,1)$}
            node[above,pos=0.65]{$x$}
            (xend);
       \coordinate (x'end) at ($(X)+(-1.1,-0.5)$);
       \draw[arr1,-,ultra thick,shorten <=4pt,transform canvas={yshift= 1pt}, atkv, dashed] (X) to (x'end);
       \draw[arr1,-,shorten <=3pt,transform canvas={yshift= -0.75pt}, defv] (X) to +(x'end);
       \draw[arr1, <<-] (X) to node[above,pos=0.65]{$x'$} (x'end);
       \coordinate (yend) at ($(Y)+(1,0.5)$);
       \coordinate (y'end) at ($(Y)+(1,-0.5)$);
       \draw[arr1,-,shorten <=6pt,transform canvas={yshift=-0.75pt}, defv] (Y) to (yend);
       \draw[arr1, <<-] (Y) to node[above,pos=0.65]{$y$} (yend);
       \draw[arr1,-,shorten <=6pt,transform canvas={yshift=0.75pt}, atkv] (Y) to (y'end);
       \draw[arr1, <<-] (Y) to node[above,pos=0.65]{$y'$} (y'end);
   \end{tikzpicture}
   $\cong$
   \begin{tikzpicture}[center base]
       \def\spray{.24}
       \node[dpad1] (X) at (0,-0.6) {$X$};
       \node[dpad1] (Y) at (1.4,0) {$Y$};
       \node[dpad1] (T) at (0, 0.6){$\Theta$};
       % \draw[arr1,-,shorten >=3pt,transform canvas={yshift=-1pt}, defv,dashed,ultra thick] (X) to (Y);
       % \draw[arr1,-,shorten >=3pt,transform canvas={yshift=0.75pt}, atkv] (X) to (Y);
       % \draw[arr1] (X) to node[above] {$p$} (Y);
       \mergearr[arr1]XTY
       \node[above right=0pt and 0pt of center-XTY]{$p$};
       \coordinate (tend) at ($(T)+(-1.1,\spray)$);
       \draw[arr1,-,ultra thick,shorten <=7pt,transform canvas={yshift= -1.0pt}, defv, dashed] (T.160) to (tend);
       \draw[arr1,-,shorten <=3pt,transform canvas={yshift= 0.7pt}, atkv] (T.160) to (tend);
       \draw[arr1, <<-] (T.160) to node[left,pos=0.9]{$\theta$} (tend);
       \coordinate (nend) at ($(T)+(-1.1,-\spray)$);
       % \draw[arr1,-,shorten <=3pt,transform canvas={yshift=-0.7pt}, defv] (T.-160) to (nend);
       \draw[arr1, <-, defv] (T.-160) to
           node[below,inner sep=2px,pos=0.7]{\scalebox{0.8}{$\mathcal N(0,1)$}} (nend);
       \coordinate (xend) at ($(X)+(-1.1,\spray)$);
       % \draw[arr1,-,shorten <=3pt,transform canvas={yshift=0.7pt}, atkv] (X.160) to (xend);
       \draw[arr1, <-, atkv] (X.160) to
           % node[above,pos=0.65]{$x$} (xend);
           node[above,inner sep=1px,pos=0.7]{\scalebox{0.8}{$\mathcal N(x,1)$}} (xend);
       \coordinate (x'end) at ($(X)+(-1.1,-\spray)$);
       \draw[arr1,-,ultra thick,shorten <=6pt,transform canvas={yshift=1.0pt}, atkv, dashed] (X.-160) to (x'end);
       \draw[arr1,-,shorten <=3pt,transform canvas={yshift= -0.7pt}, defv] (X.-160) to (x'end);
       \draw[arr1, <<-] (X.-160) to node[left,pos=0.9]{$x'$} (x'end);
       \coordinate (yend) at ($(Y)+(1,\spray)$);
       \coordinate (y'end) at ($(Y)+(1,-\spray)$);
       % \draw[arr1,-,shorten <=3pt,transform canvas={yshift=-0.7pt}, defv] (Y.35) to (yend);
       \draw[arr1, <<-,defv] (Y.35) to node[above,pos=0.65]{$y$} (yend);
       % \draw[arr1,-,shorten <=3pt,transform canvas={yshift=0.7pt}, atkv] (Y.-35) to (y'end);
       \draw[arr1, <<-,atkv] (Y.-35) to node[below,pos=0.65]{$y'$} (y'end);
   \end{tikzpicture}
   \caption{Two equivalent illustrations of adversarial training.}
   \label{fig:adv}
\end{figure}
\textbf{Adversarial training.}
% Now, include in the mutable state
% Let's now extend our mutible state to include
% Now consider an extended PDG
Suppose we want to slightly alter $x$ to obtain $x'$ that is classified as $y'$ instead of $y$.
Adding $x'$ and $y'$ to $\dg M$ and relaxing $\p_x$ to be a Gaussian centered $x$ rather than a point mass,
we get the PPDG on the left of \cref{fig:adv}.
% \[
% \begin{tikzpicture}[center base]
%     % \node[dpad1] (X) at (0,1) {$X$};
%     % \node[dpad1] (X') at (0,0) {$X'$};
%     \node[dpad1] (X) at (0,0) {$X$};
%     \node[dpad1] (Y) at (2,0) {$Y$};
%
%     \draw[arr1,-,shorten >=3pt,transform canvas={yshift=-1pt}, blue,dashed,ultra thick] (X) to (Y);
%     \draw[arr1,-,shorten >=3pt,transform canvas={yshift=0.75pt}, red] (X) to (Y);
%     \draw[arr1] (X) to node[above] {$p$} (Y);
%     % \draw[arr1] (X) to node[right] {$\mathcal N$} (X');
%     % \node[dpad1] (T) at (1, 0.7){$\Theta$};
%     % \cmergearr[arr1]XTY{1.2,0}
%     % \draw[arr1, <<-] (T) to node[above,pos=0.65]{$\theta$} +(-1,0);
%     \coordinate (xend) at ($(X)+(-1.1,0.5)$);
%     \draw[arr1,-,shorten <=2pt,transform canvas={yshift=0.75pt}, red] (X) to (xend);
%     \draw[arr1, <-] (X) to
%         %  node[left,pos=0.85]{$\mathcal N(x,1)$}
%          node[above,pos=0.65]{$x$}
%          (xend);
%     \coordinate (x'end) at ($(X)+(-1.1,-0.5)$);
%     \draw[arr1,-,ultra thick,shorten <=4pt,transform canvas={yshift= 1pt}, red, dashed] (X) to (x'end);
%     \draw[arr1,-,shorten <=4pt,transform canvas={yshift= -0.75pt}, blue] (X) to +(x'end);
%     \draw[arr1, <<-] (X) to node[above,pos=0.65]{$x'$} (x'end);
%     \coordinate (yend) at ($(Y)+(1,0.5)$);
%     \coordinate (y'end) at ($(Y)+(1,-0.5)$);
%     \draw[arr1,-,shorten <=2pt,transform canvas={yshift=-0.75pt}, blue] (Y) to (yend);
%     \draw[arr1, <<-] (Y) to node[above,pos=0.65]{$y$} (yend);
%     \draw[arr1,-,shorten <=2pt,transform canvas={yshift=0.75pt}, red] (Y) to (y'end);
%     \draw[arr1, <<-] (Y) to node[above,pos=0.65]{$y'$} (y'end);
% \end{tikzpicture}
% \]
A LIR iteration whose focus consists of the edges marked in green (with control over the dashed green edge)
   is an adversarial attack with Euclidean distance \cite{biggio2013advattk}.
% Taking the blue view, on the other hand,
The blue focus, by contrast, ``patches'' the adversarial example by
   adjusting the model parameters to again classify it correctly.
Thus, LIR that alternates between the two and refreshes $(x,y,y')$ is adversarial training, a standard defense to adversarial attacks \cite{goodfellow2014explaining}.
%
Thus, it is just as sensible to train the inputs, as the network \cite{FNNS}.

% \textbf{The Adam Optimizer.}
% \oliver{I think this should be straightforward and powerful addition: we just need to use the control mask to give the appropriate adaptive learning rates. }
% \mehran{How would you retain memory for momentum and exponential moving average of gradient norms?}

% (S, right.)
% \[
%     \begin{tikzpicture}[center base,atkv/.style={red!50!white},defv/.style={green!50!gray}]
%         \def\spray{.24}
%         \node[dpad1] (X) at (0,0) {$X$};
%         \node[dpad1] (Y) at (2,0.5) {$Y$};
%         \node[dpad1] (T) at (0, 1){$\Theta$};
%         % \draw[arr1,-,shorten >=3pt,transform canvas={yshift=-1pt}, defv,dashed,ultra thick] (X) to (Y);
%         % \draw[arr1,-,shorten >=3pt,transform canvas={yshift=0.75pt}, atkv] (X) to (Y);
%         % \draw[arr1] (X) to node[above] {$p$} (Y);
%         \mergearr[arr1]XTY
%         \coordinate (tend) at ($(T)+(-1.1,\spray)$);
%         \draw[arr1,-,ultra thick,shorten <=3pt,transform canvas={yshift= -1.5pt}, defv, dashed] (T.160) to (tend);
%         \draw[arr1,-,shorten <=3pt,transform canvas={yshift= 1.0pt}, atkv] (T.160) to (tend);
%         \draw[arr1, <<-] (T.160) to node[above,pos=0.65]{$\theta$} (tend);
%         \coordinate (nend) at ($(T)+(-1.1,-\spray)$);
%         % \draw[arr1,-,shorten <=3pt,transform canvas={yshift=-0.7pt}, defv] (T.-160) to (nend);
%         \draw[arr1, <-, defv] (T.-160) to node[left,pos=0.95]{$\mathcal N(0,1)$} (nend);
%         \coordinate (xend) at ($(X)+(-1.1,\spray)$);
%         % \draw[arr1,-,shorten <=3pt,transform canvas={yshift=0.7pt}, atkv] (X.160) to (xend);
%         \draw[arr1, <-, atkv] (X.160) to
%             % node[above,pos=0.65]{$x$} (xend);
%             node[left,pos=0.95]{$\mathcal N(x,1)$} (xend);
%         \coordinate (x'end) at ($(X)+(-1.1,-\spray)$);
%         \draw[arr1,-,ultra thick,shorten <=3pt,transform canvas={yshift=2pt}, atkv, dashed] (X.-160) to (x'end);
%         \draw[arr1,-,shorten <=3pt,transform canvas={yshift= -1.0pt}, defv] (X.-160) to (x'end);
%         \draw[arr1, <<-] (X.-160) to node[below,pos=0.65]{$x'$} (x'end);
%         \coordinate (yend) at ($(Y)+(1,\spray)$);
%         \coordinate (y'end) at ($(Y)+(1,-\spray)$);
%         % \draw[arr1,-,shorten <=3pt,transform canvas={yshift=-0.7pt}, defv] (Y.35) to (yend);
%         \draw[arr1, <<-,defv] (Y.35) to node[above,pos=0.65]{$y$} (yend);
%         % \draw[arr1,-,shorten <=3pt,transform canvas={yshift=0.7pt}, atkv] (Y.-35) to (y'end);
%         \draw[arr1, <<-,atkv] (Y.-35) to node[below,pos=0.65]{$y'$} (y'end);
%     \end{tikzpicture}
% \]


% \section{The EM Algorithm (and VI) as LIR}
% \section{The EM Algorithm and Variational Inference as LIR}
\section{The EM Algorithm as LIR}
Suppose we have a generative model $p(Z,X | \Theta)$
describing the probability over an observable variable $X$ and a latent one $Z$.
% Let $x$ be an observation of $X$, as before.
%
Given an observation $X{=}x$,
how can we learn model parameters $\theta$ despite the missing data $z$?
There is an overwhelmingly standard answer that has independently arisen in many different domains:
the EM algorithm \citep{dempster1977maximum,mclachlan2008algorithm}.
% It iteratively computes
% So called because it
% The name of the EM is descriptive:
The EM algorithm iteratively improves an estimate of $\theta$ as follows: first estimate an 
\textbf Expectation over the missing values $z$ (fixing the current estimate $\theta$), and then perform a 
\textbf Maximization over $\theta$ to update the parameters;
% so altogether it iteratively computes
altogether this amounts to computing:
\[
   \theta^{(t+1)}_{\text{EM}}
           := \arg\max_{\theta}\; \Ex\nolimits_{z\sim p(Z|x,\theta_{\text{EM}}^{(t)})} \big[ \,\log p(x, z | \theta) \,\big].
\]
% \[
%     \dg M := ~~
%     \begin{tikzpicture}[center base]
%         \node[dpad1] (Z) {$Z$};
%         \node[dpad1,above=.3 of Z] (X) {$X$};
%         \coordinate (A) at ($ (X)!.5!(Z) + (-0.7,0)$);
%         \node[dpad1,left=.5 of A] (T) {$\Theta$};
%         % \coordinate (A) at ($ (X)!.5!(Z) + (-1.1,0)$);
%         \draw[arr1] (T) -- node[above]{$p$} (A) -- (X);
%         \draw[arr1] (T) -- (A) -- (Z);
% %
%         \draw[arr2, <<-] (X) --  node[above,pos=0.8]{$ x$} ++(0.9, 0);

%         \begin{scope}[red]
%             \draw[arr2, <<-] (T) --  node[above,pos=0.8]{$\theta$} ++(-0.9, 0);
%             \draw[arr2, <-] (Z) --  node[above,pos=0.8]{$q$} ++(0.9, 0);
%         \end{scope}
% % 			\draw[arr2, <-] (Z) -- node[above,pos=0.6]{$ q^{\{\beta =\infty\}}$} ++(-0.9, 0);%
%         % \ar[r,"p"] \& Z \ar[r,"p", bend left] \& X \ar[l,"q", bend left] \& \ar[l, two heads, "x"']
%     \end{tikzpicture}
% \]

\begin{linked}{proposition}{LIR}
   % If there are no local minima,
   % If $\theta \mapsto p_\theta(x,Z)$ is log-convex,
   %  then
   % Under mild assumptions,
   % alternating between focus $\theta$ and focus $q$ with $\chi=\infty$
   % LIR on the PDG
   % \[
   \textsc{LIR}$\Bigg(\!
       \begin{tikzpicture}[center base]
           \node[dpad0] (Z) {$Z$};
           \node[dpad0,left=.5 of Z] (X) {$X$};
           \draw[arr2, <<-] (X) --  node[above,pos=0.8]{$ x$} +(-0.9, 0);
           \coordinate (A) at ($ (X)!.5!(Z) + (0,0.8)$);
           \draw[arr1] (A) -- node[left, inner sep=3pt]{$p$} ++(0,-0.35) -- (Z);
           \draw[arr1] (A) -- ++(0,-0.35) -- (X);
%
           % \draw[arr2, <<-] (X) --  node[above,pos=0.8]{$ x$} ++(0.9, 0);
           \draw[arr2, <-] (Z) --
               node[above,pos=0.65, inner sep=2pt]{$q$}
               node[below,pos=0.7, inner sep=2pt]{${\color{gray}\scriptscriptstyle(\infty)}$}
               ++(0.9, 0);%
           %\scriptstyle q^{\{\beta =\infty\}}
           % \ar[r,"p"] \& Z \ar[r,"p", bend left] \& X \ar[l,"q", bend left] \& \ar[l, two heads, "x"']
       \end{tikzpicture}
   \!\Bigg)$
       % \]
   % in which {\sc Refocus} fixes $\varphi, \gamma = \mat 1$
   in which {\sc Refocus} re-samples $x$ 
   and alternates between
   % $\chi = \infty \mathbbm1_{\{\theta\}}$ and $\mathbbm1_{\{q\}}$
   full control of $p$ and $q$
   implements EM, in that
   $\theta_{\text{EM}}^{(t)} = \theta_{\text{LIR}}^{(2t)}$
   % for all $t \in \mathbb N$
   .
   % $ \theta^{(t+1)}
\end{linked}

% At a technical level, this is essentially a restatement of a theorem
% This result was first observed by
% A result like this was first observed by
This result is closely related to one due to
% This result is a close relative of a due to
\citet{neal1998view},
who view it as an intuitive explanation of why the EM
algorithm works.  Indeed, it is obvious in this form that
every adjustement reduces the overall inconsistency.
The result can also be readily adapted to an entire dataset by replacing $x$ with a high confidence emperical distribution, or batched with the same technique in \cref{sec:SGD}.
% It also captures fractional EM when $\chi < \infty$.
When the M step does not fully solve the maximization problem, but merely makes a 
% (\citeyear{neal1998view})
% By monotonicity, it is also clear that

\paragraph{Variational inference as LIR.}
This form of the EM algorithm is closely related to variational inference.
Indeed, analogous choices applied to the analysis of \citet{one-true-loss}
% to obtain a concrete training algorithm for variational autoencoders.
yields the usual training algorithm for variational autoencoders (VAEs). Using our language, a VAE is 


\emph{Mean-field} variational methods are the ones that  


\section{Generative Adversarial Training as LIR}
\def\pdata{p_{\mathrm{data}}}
\def\real{{\mathrm{real}}}
\def\fake{{\mathrm{fake}}}
LIR also subsumes more complex training procedures such as the one used to train
Generative Adversarial Networks, or GANs \cite{goodfellow2020generative}.
% The general idea is that
The goal is to train a generator network $G$ to generate images that cannot be distinguished
   from real ones.
%
% In a GAN, we have a generator $G$ that represents a distribution over fake images,
% an empirical distribution of real images, and a discriminator that
% aims to classify images as real or fake.
% The training process involves a second network,
Formally, let $X$ be either a generated image $X_{\fake}\sim G$ or one from a dataset
$X_\real \sim \pdata$, based on the outcome of a fair coin $C$.
A discriminator $D$ then predicts the outcome $C$ of the coin flip from the image $X$.
%
% Consider the PDG
% Finally, the generator believes that $C$ is not determined by $X$,
%     which can be captured with a purely qualitative edge with $\alpha=-1$.
% These components are summarized by the PDG
The only additional ingredient we will need that is not explicit in the original formulation is a belief $e$ that the coin is equally likely heads as tails given $X$---intuitively, representing the goal of the generator and the opposite of the goal of the discriminator.
% state of affairs is summarized below.
All of these pieces of probabilistic information can be summarized by the PDG below.
% , where ``!'' means $\beta=\infty$, i.e, full confidence.
%
% \vspace{-1ex}
\[
   % \dg M:=
   \dg M(\Theta) :=
   \begin{tikzpicture}[center base,Dcolor/.style={green!70!black},Gcolor/.style={blue}]
       % \node[dpad1](Z) at (0, 0.6){$Z$};
       \node[dpad1](Xfake) at (0,1) {$X_{\fake}$};
       \node[dpad1](Xreal) at (0,0){$X_{\real}$};
       \node[dpad1](X) at (2.1,0){$X$};
       % \node[dpad1](C) at (2,1){Real / Fake};
       % \node[dpad1](C) at (2.5,1.1){$\mathrm{Real / Fake}$};
       \node[dpad1](C) at (2.4,1.0){$C$};

       % \draw[arr, <-] (Z) to node[above,pos=0.6]{$\cal N!$} +(-1.2,0);
       %%%%% G %%%%%%
       \draw[arr1,-,line width=1.3pt,transform canvas={yshift=1pt},dashed,Gcolor, shorten <=5pt] (Xfake) to +(-1.5,0);
       \draw[arr1, <-] (Xfake) to node[above,pos=0.6]{$G$}
           % node[below,pos=0.7,inner sep=1.5pt]{\color{gray}\scriptsize$(\beta{=}1)$}
           +(-1.5,0);
       \draw[arr1, <-] (Xreal) to node[above,pos=0.6]{$\pdata!$} +(-2,0);
       % \draw[arr] (X.0) to[out=0,in=-80,looseness=1.6] node[right]{$D$} (C.-35);
       % \draw[arr] (X.10) to[out=10,in=-90,looseness=1.4] node[left]{$\frac12$} (C.-45);
       %%%%%%%%%%%%%%%%%%%%%%%%%%%%% D %%%%%%%%%%%%%%%%%%
       % \draw[arr,-,line width=1.3pt,transform canvas={xshift=1pt}, Dcolor,dashed]
       %            (X.-2) to[out=0,in=-20,looseness=2.21] (C.-28);
       \draw[arr] (X.0) to[out=0,in=-20,looseness=2.2] node[right]{$D$} (C.-30);
       % \draw[arr] (X.10) to[out=10,in=-45,looseness=1.6] node[left]{$\frac12$} (C.-45);
       %%%%%%%%%%%%%%%%%%%%%%%%%%% e %%%%%%%%%%%%%%%%%%%%
       % \draw[arr1,-,line width=1.1pt,transform canvas={xshift=0.5pt}, Gcolor,shorten >=3pt]
       %     (X.10) to[out=10,in=-55,looseness=1.6] (C.-50);
       % \draw[arr1,-,line width=1.1pt,transform canvas={xshift=-0.8pt}, Dcolor,shorten >=3pt,shorten <=1pt]
       %     (X.10) to[out=10,in=-55,looseness=1.6] (C.-50);
       \draw[arr1]
           % (X)
           (X.10)
           to[out=10,in=-55,looseness=1.6]
           % node{\scalebox{0.8}{\contour{white}{\small$50/50$}}}
           node[left,pos=0.6,inner sep=4pt]{$e$}
           (C.-50);
       % \draw[arr] (X) to[] node[]{\small50/50} (C);
       % % \cmergearr[->>]{Xfake}{Xreal}{X}{1.2,0.1}
       \coordinate (ctr) at (1.2,0.1);
       \draw[arr1,-,shorten >=0pt] (Xfake) to[bend right=8] (ctr);
       \draw[arr1,-,shorten >=0pt] (Xreal) to[bend left=5] (ctr);
       \draw[arr1,-,shorten >=0pt] (C) to[in=150,out=180,looseness=1.5](ctr);
       \draw[arr1,->>,shorten <=0pt] (ctr) to[bend right=10](X);
       \draw[arr1, <-] (C) to node[above,pos=0.6]{\small$50/50!$} +(1.5,0);
\end{tikzpicture}
\]
GAN training is often written as a 2-player game
$
   \min_{G} \max_{D}  \mathcal L^{\text{GAN}}(G,D)
$, where
\[
\mathcal L^{\text{GAN}}(G,D) = 
   % \Ex_{\vphantom{|}x \mathrlap{\sim \pdata}}
   \Ex\nolimits_{x \sim \pdata}
   [\,\log D(x)\,] + \Ex\nolimits_{x' \sim G} [\,\log (1- D(x'))\,].
\]
% Now consider the two perspectives: the generator's perspective $G$, with
% \textbf{The Discriminator's View.}
% % According to the discriminator,
% \textbf{The Generator's View.}
% The generator controls t

\textbf{The Discriminator's Focus.}
The discriminator has full control over $D$, and attends to
everything but $e$.
That inconsistency of this PDG is what might be called
the discriminator's objective:
% the KL divergen
% $\Ex[ \kldiv{D(C|X)}]$
the expected KL divergence from $D$ to the optimal discriminator.
If $D$ also disbelieves that any image is equally likely to be fake as real
(by chosing $\varphi(e) = -1$),
% (i.e., $\begin{matrix}\varphi(e) = -1\\ \varphi(D) =+1\end{matrix}$),
then the inconsistency becomes $-\mathcal L^{\text{GAN}}$.
% $D(C|X)$ to the
% true probabililty of $C$ given $X$.
% while the generator has control over $G$.

\textbf{The Generator's Focus.}
The generator has control over $G$.
If it ignores $D$ attends only to $e$, the inconsistency
is the Jensen-Shannon Divergence between $G$ and $\pdata$.
If the generator also disbelieves the discriminator $D$
(i.e., $ \varphi(D) =-1$),
then the inconsistency becomes $+\mathcal L^{\text{GAN}}$.

Standard practice is to use small $\chi(G)$ and large $\chi(D)$.

\section{Message Passing Algorithms as LIR}
   \label{sec:factor-graph}

% \paragraph{Belief Propogation.}
% % Let $\X$ be a set of variables,
% Consider a factor graph
A \emph{factor graph} over a set of variables $\X$ is a set of factors
% Let $\X$ be a set of variables, and consider a set of factors
$\Phi = \{ \phi_a : \mathbf X_a \to \mathbb R_{\ge 0}\}_{a \in \Ar}$,
where each $\mathbf X_a \subseteq \mathcal X$ is called the \emph{scope} of $a$.
Conversely, for $X \in \X$, let
$\partial X
    := \{a \in \Ar : X \in \mat X_a\}
$ be the set of factors with $X$ in scope.
The factor graph
$\Phi$ specifies a distribution
$\Pr_\Phi(\X) \propto \prod_a \phi_a(\mat X_a)$, and
% $\Pr_\Phi = \frac1Z \prod_a \phi_a(\mat X_a)$, where $Z$ is a normalization constant,and $\log Z$ is  and
corresponds to a PDG
%
% The PDG that corresponds to
% this factor graph is $\Phi$ is
\[
% $
   \dg M_\Phi = \Big\{ \begin{tikzpicture}[center base]
       \node[dpadinline] (X) {$\mathbf X_a$};
       \draw[arr,<-] (X) to node[above, pos=0.65,inner sep=3pt]{$\propto \smash{\phi_a}$}
           node[below,pos=0.7,inner sep=1.5pt]{\color{gray}\scriptsize$(\alpha,\beta{=}1)$} +(-1.4,0);
   \end{tikzpicture}~\Big\}_{a \in \Ar}
% $
\]
% ~\parbox{0.45\linewidth}{
% \!\!
that specifies the same distribution $\Pr_\Phi(\X)$ when $\gamma=1$.
% }
% be the corresponding PDG.
% There are several different variants of beleif propogation;
% each can be written as an instance of LIR.

% \paragraph{Sum Product Belief Propogation.}
% The state for standard sum-product belief
Sum-product belief propogation \cite{kschischang2001factor}
   aims to approximate marginals of $\Pr_{\Phi}$
   with only local computations: messages sent between factors and
       the variables they have in scope.
% passing messages between adjacent variables and factors.
% so the state of the algorithm consists of a pair of messages
Its state consists ``messages''
$m_{X \sto a}$ and $m_{a \sto X}$
% $\{ m_{X \sto a}, m_{a \sto X}\}$, 
both (unnormalized) distributions over $X$,
for each variable $X$ and factor $a \in \partial X$ adjacent to it.
% with $a \in \partial X$
% for each pair $(X, a \in \partial X)$.
% $\nu_X(X)$ for each variable $X$, and a joint distribution $\nu_a(\mat X_a)$ for each
% factor $a \in \Ar$.
% $m_{i}
% Concretely, the messages passed are given by:
% Together, we can regard this collection of messages as a PDG, which we call $\Msg$.
% Just as with the original factor graph,
   % $\Msg$.
% which together form a PDG that we will call $\Msg$
% in the same way as
% the original factor graph.
After initialization, belief propogation repeatedly recomputes:
% The bulk of the algorithm involves propogating messages, each of which
% is a function of other messages, as follows:
\begin{align}
   m_{X \!\sto a}(x)
       % &:=
       &:\propto
       \prod_{{b \in \partial X\setminus a}} m_{b\sto\! X} (x)
       % \qquad &
       \label{eq:X->a}
       \\
   m_{a \sto\mskip-2mu X}(x)
       % &:=
       &:\propto
       ~~\sum_{\mathclap{\mat y \in \V(\mat X_a \setminus X)}}~~ \phi_a(x, \mat y)
       ~\prod_{\mathclap{Y \in \mat X_a \setminus X}}~
               m_{Y \!\sto a} (Y(\mat y)),
       \label{eq:a->X}
\end{align}
where $Y(\mat y)$ is the value of $Y$ in the joint setting $\mat y$.
Observe that every calculation is a (marginal of) a product of factors,
   and thus amounts to inference
       % on a local factor graph of the appropriate context.
       in a ``local'' factor graph.

% On the factor graph, each node aggregates
% The process can be visualized as propogation along the edges of the factor graph.
% The standard illustration of it is not so different from the PDG $\Msg$
The usual schematic illustration people draw to depict messages
   moving between variables and factors according to \cref{eq:X->a,eq:a->X}
   is as follows:
\begin{center}
   \begin{tikzpicture}[xscale=2.1,yscale=1.3]
       \node[factor] (a) at (0,0) {$a$};
       \node[draw,circle,inner sep=4px] (X) at (1,0) {$X$};
       \node[factor] (b1) at (2,0.5){$b_1$};
       \node[factor] (bm) at (2,-0.5){$b_m$};
       \node[draw,circle,inner sep=2px] (Y1) at (-1,0.5){$Y_1$};
       \node[draw,circle,inner sep=2px] (Yn) at (-1,-0.5){$Y_n$};
       % \node[anchor=center] at (-1,0){$\scalebox{1}[1]{\vdots}$};
       % \node[anchor=center,draw] at (2,0){$\scalebox{1}[1]{\vdots}$};
       \draw (a) -- (X);
       \draw (b1) -- (X);
       \draw (bm) -- (X);
       \draw (Y1) -- (a);
       \draw (Yn) -- (a);

       % \draw[arr0,red,arrows={->[harpoon,swap]}]
       %     (1.6,0.7) to[in=23,out=-80] node[left, pos=0]{$m_{b_1 \!\sto \!X}$\!\!} (X);
       % \draw[arr0,red,arrows={->[harpoon,swap]},dashed]
       %     (0.5,0.7) to[in=0,out=-90] node[left, pos=0]{$m_{X \!\sto a}$\!} (a);
       % \draw[arr0,blue,arrows={->[harpoon,swap]}]
       %     (0.5,-0.7) to[in=180,out=90] node[right, pos=0]{\!$m_{a \sto \!X}$} (X);
       \begin{scope}[transform canvas={yshift=2px},green!70!black]
       \draw[arr,arrows={->[harpoon,swap]}]
           (b1) to node[above=-0.5pt,rotate=15]{$m_{b_1 \!\sto \mskip-2muX}$\!\!} (X);
       \draw[arr,arrows={->[harpoon,swap]}]
           (bm) to node[above=-0.5pt,rotate=-5,pos=0.3]{$m_{b_m \!\sto \mskip-2muX}$\!\!} (X);
       \draw[arr,arrows={->[harpoon,swap]},densely dotted]
           (X) to node[above]{$m_{X \!\sto a}$\!} (a);
       \end{scope}

       \begin{scope}[transform canvas={yshift=-2px},blue]
       \draw[arr,arrows={->[harpoon,swap]}]
       % \draw[blue,arrows={->[harpoon,swap]}]
           (Y1) to node[below=-0.5pt, rotate=-5,pos=0.35]{$m_{Y_{\!1} \!\sto a}$\!\!} (a);
       \draw[arr,arrows={->[harpoon,swap]}]
           (Yn) to node[below, rotate=17]{$m_{Y_{\!n} \!\sto a}$\!\!} (a);
       \draw[arr,arrows={->[harpoon,swap]},densely dotted]
           (a) to node[below]{\!$m_{a \sto \mskip-2muX}$} (X);
       \end{scope}
       % \draw (current bounding box.north west) rectangle (current bounding box.south east);
   \end{tikzpicture}~\raisebox{1ex},
\end{center}%

While this standard diagram is only a schematic,
   simply writing down the messages as unconditional probability
   distributions, we get a PDG $\Msg$ that can be made to look very similar.
Formally, a variable $X^{a}$ for every pair
   % $(a \in \Ar, X \in \mat X_a)$,
   $(X,a)$ with $X \in \mat X_a$
along with edges asserting that $X^{a} = X$,
we obtain the equivalent PDG 

% (see \cref{sec:bp-details}) is not so different from the PDG
% \begin{quotation}\it
%     the message sent from a vertex $v$ on an edge $e$ is the product of the local function at $v$, with all messages recieved at $v$ (other than from $e$),
%     summarized for $e$'s variable node. \cite{kschischang2001sumproduct}
% \end{quotation}
% \vskip
% The PDG $\Msg$ is not so different from this schematic:
% \begin{center}
\[
   %  \Msg+
   % {\color{blue!50!black}\dg M_\Phi}
   % = 
   % \sum_{a \in \Ar} \sum_{X \in \mat X_a} \quad
   % $\begin{matrix}\Msg\\+
   %     {\color{blue!50!black}\dg M_\Phi}
   % \end{matrix}\!\cong\,$
   \begin{tikzpicture}[xscale=2,center base]
       % \node[factor] (a) at (0,0) {$a$};
   \begin{scope}[os1/.style={outer sep=1pt},dpad0/.append style={fill=gray!50!black,text=white,font=\mathversion{bold}}]
       % \node[] (a) at (0,0) {$a$};
       % \node[dpad0,os1] (aX) at (0.22,0){\scalebox{0.7}{$X^{(a)}$}};
       \node[dpad0,os1] (aX) at (0.22,0){\scalebox{0.7}{$X^{a}$}};
       \node[dpadded,os1] (X) at (1,0) {$X$};
       % \node[factor] (b1) at (2,0.5){$b_1$};
       % \node[factor] (bm) at (2,-0.5){$b_m$};
       \node[dpad0,os1] (b1X) at (1.8,0.5){\scalebox{0.7}{$X^{b_1}$}};
       \node[dpad0,os1] (bmX) at (1.8,-0.5){\scalebox{0.7}{$X^{b_m}$}};
       % \node[draw,circle,inner sep=2px] (Y1) at (-1,0.5){$Y_1$};
       % \node[draw,circle,inner sep=2px] (Yn) at (-1,-0.5){$Y_n$};
       \node[dpadded,os1] (Y1) at (-1.1,0.5){$Y_1$};
       \node[dpadded,os1] (Yn) at (-1.1,-0.5){$Y_n$};
       % \node[dpad0,os1] (aY1) at (-0.2,0.3){\scalebox{0.7}{$Y_{1}^{(a)}$}};
       % \node[dpad0,os1] (aYn) at (-0.2,-0.3){\scalebox{0.7}{$Y_{n}^{(a)}$}};
       \node[dpad0,os1] (aY1) at (-0.2,0.3){\scalebox{0.7}{$Y_{1}^{a}$}};
       \node[dpad0,os1] (aYn) at (-0.2,-0.3){\scalebox{0.7}{$Y_{n}^{a}$}};
   \end{scope}
       % \node[anchor=center] at (-1,0){$\scalebox{1}[1]{\vdots}$};
       % \node[anchor=center,draw] at (2,0){$\scalebox{1}[1]{\vdots}$};
       % \draw (a) -- (X);
       % \draw[double equal sign distance,shorten <=0,shorten >=0] (aX.center) to (X.center);
       % \draw[double equal sign distance,shorten <=0,shorten >=0] (aX) to (X);
   \begin{scope}[every path/.append style={gray!50,double equal sign distance}]
       \draw (aX) to (X);
       \draw (aY1) to (Y1);
       \draw (aYn) to (Yn);
       \draw (b1X) to (X);
       \draw (bmX) to (X);
   \end{scope}
       % \draw (b1) -- (X);
       % \draw (bm) -- (X);
       % \draw (Y1) -- (a);
       % \draw (Yn) -- (a);

       % red arrows X->a
   \begin{scope}[green!70!black]
       \draw[arr0,
           % arrows={->[harpoon,swap]}
           ]
           (1.5,0.8) to[in=20,out=-80] node[left, pos=0.1]{$m_{b_1 \!\sto \mskip-2muX}$\!} ([yshift=1px]X.24);
       \draw[arr0,
           % arrows={->[harpoon,swap]}
           ]
           (1.6,0.1) to[in=-23,out=-95] node[right, pos=0.25]{$m_{b_m \!\sto \mskip-2muX}$\!} ([yshift=1px]X.-15);
       \draw[arr0,
           % arrows={->[harpoon,swap]},
               dashed]
           (0.6,0.7) to[in=0,out=-90] node[left, pos=0.1]{$m_{X \!\sto a}$\!} ([yshift=2px]aX.east);
   \end{scope}
       % blue arrows a -> X
   \begin{scope}[blue]
       \draw[arr0,
           % arrows={->[harpoon,swap]}
           ]
           (-0.85,-0.1) to[in=160,out=110] node[right, pos=0.1]{$m_{Y_1 \!\sto a}$\!} ([yshift=-2px]aY1.175);
       \draw[arr0,
           % arrows={->[harpoon,swap]}
           ]
           (-0.8,-0.9) to[in=-157,out=100] node[right, pos=0.1]{$m_{Y_n \!\sto a}$\!} ([yshift=-1px]aYn.-166);

       \draw[arr0,
           % arrows={->[harpoon,swap]},
           dashed]
           (0.6,-0.7) to[in=180,out=90] node[right, pos=0.1]{\!$m_{a \sto \mskip-2muX}$} ([yshift=-2px]X.west);
   \end{scope}
   \begin{scope}[blue!50!black]
       \coordinate (amerge) at (-70:0.6);
       \draw[arr0,shorten <=0] (amerge) to[out=130,in=-90] (aX);
       \draw[arr0,shorten <=0] (amerge) to[out=130,in=-45] (aY1);
       \draw[arr0,shorten <=0] (amerge) to[out=130,in=-10] (aYn);
       \draw[arr0,-,shorten >=0] (0.3,-0.9) to[in=-50,out=90]
           node[left]{$\phi_a$} (amerge);
   \end{scope}
\end{tikzpicture}
\raisebox{-1em}{
\!\!
$\subseteq\begin{matrix}\Msg\\+
   {\color{blue!50!black}\dg M_\Phi}
   \text{.\!}
\end{matrix}$}
% \end{center}
\]
%
% Computing
% Calculating
% $m_{X \sto a}$
% the message from $X$ to $a$
% Our observation here is that the message computations are
Indeed, it can be shown that (\ref{eq:X->a},\ref{eq:a->X}) minimize inconsistency of
   the dotted components in their appropriate contexts
   (shown in green and blue above, and formalized
       in \cref{sec:bp-details}).
   % shown above in orange and blue, respectively.
% Once the details are in place, we find that

Finally, variable marginals $\{b_X\}_{X \in \X}$,
which we regard as another PDG, $\dg B$, are computed from the messages according to
$
   b_X(x) \propto \prod_{a \in \partial X} m_{a \sto X}(x)
$.

% Thus, for any schedule of messages
\begin{linked}{proposition}{bp}
   If {\sc Refocus} selects a focus non-deterministically from
   $\{ a\sto\mskip-2mu X, X\! \sto a, X\}_{X \in \X, a \in \partial X}$
   (see illustration above; details in \cref{sec:bp-details}), then
   % with attention and control sets as above and
   % $\gamma=1$, then
   % \[ \Big\{
   %     \begin{pmatrix}
   %         % \chi = \infty \mathbbm 1[m_{X \sto a}]\\
   %         % \phi = \infty\mathbbm1 [\{ m_{b \sto X} \}_{b \in \partial X \setminus a} \cup \{m_{X \sto a}\}]
   %         C =  \{ m_{X \sto a} \}\\
   %         A = \{ m_{b \sto X} \}_{b \in \partial X \setminus a} \cup \{m_{X \sto a}\}
   %     \end{pmatrix}
   %     ~\Big|~ a \in \Ar, X \in \mat X_a \Big\},
   % \]
   the possible runs of
   {\sc LIR}$(
       \dg M_\Phi, \Msg
       + \dg B
        )$
   are precisely those of BP for different message schedules.
   % Further adopting a view with full control in $\beta$
\end{linked}

The same construction works for general cluster graphs,
   and we suspect the same is true of the many other message passing algorithms
   that are generated from the $\alpha$-divergences \cite{minka2005divergence},
   because of the close relationship those divergences have with simple PDGs
   \cite[\S5]{one-true-loss}.
% generalized belief propogation
   % \cite{generalized-bp} as well.

% \paragraph{Message Passing from Divergences}
% \citet{}
% \section{Adversarial Training as LIR}

% we also believe

%
% Here, various control comes in handy.
% \section{Diffusion as LIR}
% The diffusion objective turns out to be just the inconsistency of the appropriate PDG,
% i.e.,
% \[
%     .
% \]
% % but it has
% What makes this tractable, however, is that

% \section{Generic Generative Modeling with LIR}
% All of this suggests a general process by which
% Suppose we want to generate some structured data .
% The general idea

\section{GFlowNet Training as LIR}


Generative Flow Networks, or GFlowNets~\cite{gflownet}, are probabilistic generative models that aim to sample from a distribution that is given in the form of unnormalized probabilities, referred to as rewards and denoted by $R$. The sample is constructed by sequentially modifying an initial sample via a prescribed set of actions $\mathcal{A}$. Let $S_t$ denote the random variable corresponding to the state of the sample after $t$ modifications. The effect of actions on states can be represented as Directed Acyclic Graph (DAG). 
Different objectives have been used to train GFlowNets, but each is an attempt to enforce flow-balancing conditions (e.g., conservation of flow, detailed balance, trajectory balance) along the DAG of the generative process. For illustration, we analyze the trajectory balance (TB) objective:
\[
\mathcal{L}^\mathrm{TB} = \Ex_{s \sim Q}\left[ \log^2\left( \frac{\prod_i P_F(s_{i+1} \mid s_i)}{\prod_i P_B(s_i \mid s_{i+1})} \right) \right]
\]
where, by definition, $P_B(s_n | s_{n+1}) = R(s_n)$
% \mehran{I think different losses have been used for GFNs. Which paper are we following? Should there be a product inside the log or a sum outside instead, i.e., a loss that is applied for each edge separately?}
% \oliver{the TB and subTB losses seem, which are closest to what is written here, seem to be most used in practice. The product inside is the standard way to write it.} \mehran{Reward is missing from the loss function...} \oliver{true; normally you'd also put $R(x)$ in the denominator and $F(s_0)$ in the numerator, but neither has an obvious probabilistic interpretation so let's revisit this later.}

The probabilistic dependency graph of a GFlowNet can be depicted as follows:
\[
\begin{tikzpicture}
   % Nodes S_0, S_1, ..., S_n
   \node[dpad1] (S0) {$S_0$};
   \node[dpad1] (S1) [right=of S0] {$S_1$};
   \node (dots) [right=of S1] {$\cdots$};
   \node[dpad1] (Sn) [right=of dots] {$S_n$};

   % Forward curved arrows
   \draw[arr1, bend left=30] (S0) to (S1);
   \draw[arr1, bend left=30] (S1) to (dots);
   \draw[arr1, bend left=30] (dots) to (Sn);

   % Backward curved arrows
   \draw[arr1, bend left=30] (Sn) to (dots);
   \draw[arr1, bend left=30] (dots) to (S1);
   \draw[arr1, bend left=30] (S1) to (S0);

   % Incoming edge into S0 from the left
   \draw[arr1] ($(S0.west)+(-0.5,0)$) -- (S0.west) node[midway, above] {$P_0$};

   % Incoming edge into Sn from the right
   \draw[arr1] ($(Sn.east)+(0.5,0)$) -- (Sn.east) node[midway, above] {$R$};

   % Double multi-arrow labeled Q! going to S0, S1, Sn
   \coordinate (A) at ($ (S0)!.5!(Sn) + (0,1.5)$);
   \draw[arr2,-,shorten <=0pt,shorten >=0pt] (A) to node[left, inner sep=3pt]{$Q!$} ++(0,0.35);
   % \draw[arr2, ->>, bend right=20] (A) to node[left, inner sep=3pt]{} ++(0,-0.35) (S0);
   % \draw[arr2, ->, bend right=20] (A) -- ++(0,-0.35) to (S0);
   % \draw[arr2, ->, bend right=20] (A) -- ++(0,-0.35) to (S1);
   % \draw[arr2, ->, bend left=20] (A) -- ++(0,-0.35) to (dots);
   % \draw[arr2, ->, bend left=20] (A) -- ++(0,-0.35) to (Sn);
   \draw[arr2] (A) to[out=-90,in=90] (S0);
   \draw[arr2] (A) to[out=-90,in=90] (S1);
   \draw[arr2] (A) to[out=-90,in=90] (Sn);
\end{tikzpicture}
\]
% \mehran{Actions are missing...}
% \oliver{this is not a problem; they're already playing their appropriate role as the supports of $P_F$ and $P_B$.}

% \mehran{I'm having a lot of doubts what I proposed in the meeting but I'll write it down anyways. A possible better approach: model the log flow of each edge as a Gaussian random variable. We have two probabilistic models for this flow that are parameterized with the same learnable parameters $\theta$. One goes backward and one goes forward and they need to be made consistent.}

The inconsistency can be written as:
\[
\Ex_{s \sim Q} \left[\log \prod_i \left( \frac{Q(s_{i+1} \mid s_i)}{P_F(s_{i+1} \mid s_i)} \right)^{\phi^{(F)}_i} \left( \frac{Q(s_i \mid s_{i+1})}{P_B(s_i \mid s_{i+1})} \right)^{\phi^{(B)}_i} \right]
\]

If we let $\phi^{(F)}_i = -\phi^{(B)}_i$ then $Q$ cancels out and the forward and backward probabilities will be pushed in opposite directions. We can then use the control mask to make sure that they match by choosing its sign and magnitude according to which of $P_F$ or $P_B$ is larger and how different they are. Specifically, we must set $\chi \propto \log\left( \frac{\prod_i P_F(s_{i+1} \mid s_i)}{\prod_i P_B(s_i \mid s_{i+1})} \right)$.




% whose design is explicitly predicated on 

\section{Decision Making with LIR}

In this section, we show how a few standard decision rules can be viewed as inconsistency minimization. 
% 
% \paragraph{Expected Utility Maximization as LIR.}
\Citet[\S 4, \S6]{oli-dissertation} established that PDGs can represent expected cost,
% albeit with some strange 
albeit by articulating some questionable probabilistic beliefs. 
% You should care about it because we've already established this construction (sections 4.2.2 and 6.8) as a way of representing something like a utility or a cost.  Yes, it may seem strange, but if you buy into the perspective of the dissertation, this is because agents with preferences are already strange.  Thus, a natural question becomes: what does inconsistency resolution mean in this context? Perhaps surprisingly, it means expected utility maximization, at least if confident in one's beliefs. 
For the moment, let us entertain the possibility that these strange probabilistic beliefs are in fact an appropriate way to encode preferences.
% (Perhaps this model is appropriate, but looks unnatural only because preferences themselves are unnatural).
% (Perhaps it looks strange because preferences themselves are strange.)
% A natural 
A natural question then emerges: what does it mean to make decisions to as to minimize inconsistency in this context? 
In this section, we show that the standard picture of rational decision making---that is, expected utility maximization---can be viewed as the pursuit of consistency in this probabilistic model. 

Suppose we are trying to choose an action (i.e., control a variable $A$). 
To do so reasonably, we need to model some context. 
Let $S$ be a variable representing the current state of the world,
and $O$ represent the outcome after taking our action. 
%
To complete the standard picture, let's further assume that 
we have 
\begin{enumerate}[nosep]
\item 
   some understanding of how our action $A$ and the state $S$ determine the outcome $O$, say in the form of a cpd $\tau(O \mid S,A)$, 
\item 
a belief $p(S)$ about the current state of the world,
and 
\item  a utility function $u : \V O \to \mathbb R$ on possible outcomes. 
\end{enumerate}
It is easy to encode this information in a PDG:

\begin{center}
   \begin{tikzpicture}[scale=0.8]
       \node[dpadded] (S) at (-0.4,0) {$S$};
       \draw[arr2, <-] (S) -- node[above]{$p$} ++(-1.5, 0); 

       \node[dpadded] (O) at (2,0) {$O$};
       \node[dpadded] (U) at (4,0) {$U$};
       \draw[arr2, ->>] (O) -- node[above, pos=0.35]{$u$}  (U);
       \node[dpadded] (A) at (0.5,1.3) {$A$};
           \mergearr[arr1,->] SAO
       \node[below=1pt of center-SAO]{$\tau$};
       % \node[dpad1] (T) at (5.5,0) {$\Tru$};
   \end{tikzpicture}~.
\end{center}

% where $\V U = \mathbb R$. 
% All that's missing is a connection 
% However, this is just a description of what
What's missing is the idea 
that a higher numerical utility is ``better'' than a lower one. 
For this, we can add a ``soft constraint'' \citep[see][\S4.2.2]{oli-dissertation} that is violated less at higher utilities than lower ones.
At a technical level, recall that this means 
(1) including the variable $\Tru$ that can technically take on values $\V \Tru = \{ \truf, \trut\}$, yet happens to always on the value $\trut$,
and  
(2) adding a cpd $b(\Tru \mid U)$ encoding a constraint violation that is more serious the lower the value of $U$. 
Let 
$\dg M_{(p, \tau, u, b)}$, or simply $\dg M$, 
represent the PDG above augmented with such a soft constraint $b$. 

Intuitively, imagine that there is a part of you that you cannot control, 
   which we will call ``Faith''.
Faith
   engages in wishful thinking: 
   she disbelieves outcomes that are undesirable,
   creating epistemic conflict if she sees outcomes of low utility. 
(Technically, Faith refers to the sub-PDG consisting of $u$, $b$, and $\Tru{=}\trut$.)
If you have no control over Faith, but still have confidence and pay attention to her,
then it turns out that selecting actions to minimize your combined inconsistency
is a decision rule that interpolates between
\begin{enumerate}[nosep,label={(\alph*)}]
   \item 
   maximizing expected utility (when $\beta_p \gg \beta_b$), and
   \item 
   maximizing maximum utility (when $\beta_p \ll \beta_b$),
\end{enumerate}
where  $\beta_p$ is your confidence in your prior probabilistic belief $p(S)$, and $\beta_b$ is the confidence of the soft constraint $b$ (i.e., your ``degree of faith'').

\begin{linked}{proposition}{eumaxmax}
   % So long as $b(\Tru{=}\trut \mid U{=}u)$ is strictly increasing in $u$, 
   Suppose $b(\Tru{=}\trut \mid U{=}u) = k \cdot \exp( u )$ for some constant $k$. 
   \begin{enumerate}[nosep]
       \item 
   If $\beta_p = \infty$ and $\beta_b < \infty$, then 
   the action(s) $a \in \V\! A$ that minimize the inconsistency
   are those that maximize expected utility.
   Formally,
   for all $\gamma \ge 0$, 
   \[
       \argmin_{a \in \V\!A} \aar[\big]{\dg M
           % _{(p,\tau,u,b)}
            + A{=}a}_\gamma 
       = \argmax_{a \in \V\! A} 
           \Ex_{\substack{s\sim p \\ o \sim \tau|s,a}} \big[ u(o) \big].
   \]
   \item 
   If $\beta_p < \infty$ and $\beta_b = \infty$, then 
   the action(s) $ a \in \V\! A$ that minimize overall inconsistency are
   those that can lead to the best possible outcome, i.e., 
   \[
       \argmin_{a \in \V\!A} \aar[\big]{\dg M
           % _{(p,\tau,u,b)}
            + A{=}a}_\gamma 
       = \argmax_{a \in \V\! A} \;
           \max_{s \in \V S} \;
           \Ex_{o \sim \tau|s,a} [ u(o) ].
   \]
   \end{enumerate}
\end{linked}

% We suspect that it may also be possible to implement other decision rules such as minimax or maximin, but like GAN training, these objectives that look like ``two-player games'' require two different focuses of LIR. 
We suspect that it may also be possible to implement other decision rules such as minimax or maximin.
Like the GAN objective however, these decision rules look like two-player games, and so will likely require two different focuses of LIR, rather than just one. 
% We leave a careful exploration of to future work.
We leave a careful exploration of this avenue to future work.

We conclude our discussion of decision theory with a high-level observation. 
One's preferences, beliefs, and actions can be jointly modeled with a PDG. 
% Suppose that the 
When that PDG is inconsistent (i.e., there is a conflict between your preferences, beliefs, and actions),
% When there is conflict between your preferences, beliefs, and actions
% (in the sense that the PDG that represents all three is inconsistent),
there are, in principle, three possible resolutions.
% You can change your action to 
You can 
resolve the inconsistency by changing your action, 
   which amounts to maximizing expected utility (\cref{proposition:eumaxmax}.1);
   this is thought of as the {rational} approach.
   % maximize your expected utility, which is thought of as the rational thing to do.
Alternatively, 
   you can change your preferences so as to become content with your current situation, 
% You can also change your preferences to become happy with how things currently are, 
   which is perhaps a more zen perspective. 
Observe that these two resolutions to the internal conflict are two halves of the famous Serenity Prayer \citep{serenity-prayer}:
\begin{quotation}\it
   \noindent
   Oh, God,
   give us the courage to change what must be altered, serenity to accept what can not be helped, and insight to know the one from the other.
\end{quotation}
That ``insight'' amounts to the choice of the control mask $\chi$. 

There is of course to resolve the inconsistency: change your beliefs $p(S)$ about what state you are in: wishful thinking. 
Unfortunately, this kind of self delusion typically only leads to more inconsistency in the future, and is seldom a good idea.  
% This is one more reason that it can be beneficial to carefully select only small local of the picture to control while resolving inconsistencies.

This example clearly illustrates that,
   in resolving inconsistencies, 
   there can be wisdom in 
   restricting control $\chi$ to a small subset of parameters
   that extends far
   beyond the computational savings of doing so.
   % can be  a computational shortcut.
   % while resolving inconsistency.




\section{Discussion and Future Work}

These examples are only the beginning.
Our initial investigations suggest that
opinion dynamics models,
the training process for diffusion models,
and much more, are all naturally captured by LIR.
%
The surprising generality of LIR begs some theoretical questions.
What assumptions are needed to prove that it reduces overall inconsistency,
   as is often the case?
   How expressive is this mode of computation?

% These results also
It also
suggests a novel approach to structured generative modeling:
haphazardly assemble a PDG with many variables, existing models, priors,
   constraints, and data of all shapes and sizes.
Then, train new models to predict variables from one another,
   using LIR (with random refocusing, say).
Is this effective?  We are excited to find out!


\bibliography{lir-refs}
\bibliographystyle{apalike}


\clearpage
\appendix
\section{Details on Belief Propogation} \label{sec:bp-details}

We now define the views.
% To a first approximation
Modulo a small subtlety,
the following is essentially true:
% In particular,
\eqref{eq:X->a}
selects
$C_{X \sto a} := \{ m_{X \sto a} \}$ so as to
minimize 1-inconsistency in  context
$A_{X \sto a} := \{ m_{b \sto X} \}_{b \in \partial X \setminus a} \cup \{m_{X \sto a}\}$,
while
% $m_{a \sto X}$
% the mesage from $a$ to $X$
\eqref{eq:a->X}
% is the one that minimizes the 1-inconsistency of
selects
$C_{a \sto X} := \{ m_{a \sto X} \}$
so as to minimize the 1-inconsistency in
context
$A_{a \sto X} := \{ \phi_a, m_{a \sto X} \} \cup \{ m_{Y \sto a} \}_{Y \in \mat X_a \setminus X}$.

The only wrinkle is that we do not attend to
   the \emph{structural} aspect of the edges $e$ that we're updating;
       i.e., we must select $\varphi$ to effectively set $\alpha_e = 0$.
Intuitively: although all of the input messages summarize causal information,
   we're trying to capture that information with a distribution.
   Thus, it's not appropriate to attend to the causal structure of the edges that we're modifying. 
Thus, for each $f \in\bigcup_{a\in \Ar,X \in \mat X_a}\{a \sto X, X \sto a, X\}$, define
a focus $(\varphi_f, \chi_f) \in \mat F$ according to
\[
   \varphi_f(a) :=  \begin{cases}
       \binom 11 & \text{ if } a \in A_{f} \setminus C_{f} \\
       \binom 10 & \text{ if } a \in C_{f} \\
       \binom 00 & \text{otherwise}
   \end{cases},
   \qquad\qquad
   \chi_f(a) := \begin{cases}
   \infty & \text{ if } a \in C_{v} \\
   0 & \text{ otherwise}.
\end{cases}
\]
where $\binom{\phi_1}{\phi_2}$ scales $\beta_a$ by $\phi_1$ and $\alpha_a$ by $\phi_2$.
With these definitions, \cref{proposition:bp} follows easily.


\onecolumn
\section{Proofs}

First, some extra details for \cref{proposition:logccave}.
By parameteriations $\mathbb P$ log-concave, we mean that, for every $a \in \Ar$, and $(s,t) \in \V(\Src a,\Tgt a)$, the function
$$
   \theta \mapsto -\log \p_a^{\,\theta}(\Tgt a =t\mid\Src a = a) \quad: \Theta_a \to [0,\infty]
$$
is convex.
This is true for many families of distributions of interest.
For example, if $\Src a, \Tgt a$ is discrete, and the cpd is parameterized
by stochastic matrices $\mat P = [p_{s,t}] \in [0,1]^{\V(\Src a, \Tgt a)}$, then
\[
   - \log \p_a^{\mat P}(\Tgt a =t | \Src a=s) = - \log (p_{s,t})
\]
which is clearly convex in $\mat P$.

To take another example: if $\p_a$ is linear Gaussian, i.e.,
$\p_a(T|S) = \mathcal N(T | \mat A s + b,  \sigma^2)$, parameterized by
$(\mat A, b, 1/\sigma^2)$, then
\begin{align*}
   - \log \p_a^{(\mat A, b, \sigma^2)}(t|s)
   &= -\frac12 \log \frac{2\pi}{\sigma^2}  + \frac12 \left(\frac{t- \mat A s + b}{\sigma}\right)^2
\end{align*}
which is convex in $(\mat A, b, \frac1{\sigma^2})$.  Now, for the proof.


% \textbf{\cref{thm:cvx}}. \textit{If $\mathbb P$ is log-concave, then
% the map $\theta \mapsto \aar*{\varphi \odot( \Ctx + \dg M(\theta))}_\gamma$ is convex.}
% \begin{proof}
\recall{proposition:logccave}
\begin{lproof} 
       \label{proof:logccave}
   By definition,
   \begin{align*}
       \aar*{\varphi \odot( \Ctx + \dg M(\theta))}_\gamma
       &= \inf_\mu 
       \Big\{ 
            \SDef_{\Ctx}
           + \OInc_{\Ctx}(\mu)
           + \SDef_{\dg M(\theta)}(\mu)
           + \OInc_{\dg M(\theta)}(\mu)  
       \Big\}.
   \end{align*}
   Only the final term actually depends on $\theta$, though. Let $F(\mu)$ capture
       the first three terms. For all of our examples, and indeed, if $\gamma$ is chosen small enough, it will be convex in $\mu$ \cite[from the proof of Proposition 3.2]{pdg-aaai}.
   Then we have
   \begin{align*}
       \aar*{\varphi \odot( \Ctx + \dg M(\theta))}_\gamma
           &= \inf_\mu  \left( F(\mu) + \Ex_\mu \left[ \sum_{\ed a ST}\beta_a \log \frac{\mu(T|S)}{\p_a(T|S)} \right] \right) \\
           &= \inf_\mu  \left( F(\mu) + \Ex_\mu \left[ \sum_{\ed a ST} \beta_a\log \frac{\mu(T|S)}{\lambda(T|S)} \right] +
           \Ex_\mu \left[\sum_{\ed a ST} \beta_a \log \frac{\lambda(T|S)}{\p_a(T|S)} \right] \right)
   \end{align*}
   The second term is then entropy (relative to the base distribution), which is
       convex in $\mu$. The first term, $F(\mu)$, is convex in $\mu$ as well, and neither depend on $\theta$. The final term is linear in $\mu$.
       Since $\mathbb P$ is log-convex in $\theta$, the $\log\frac{\lambda(t|s)}{\p_a(t|s)}$ convex in $\theta$, and so that third term is a conic combination of expectations that are all convex, and hence itself convex in $\theta$.
       Thus, the sum of all three terms in the infemum is jointly convex in $\theta$ and in $\mu$. Taking an infemum over $\mu$ pointwise, the result is still convex in $\theta$.
   % Since $\varphi \ge 0$ and $\bbeta  > 0$ this is effectively just the inconsistency of a PDG.
\end{lproof}


\recall{proposition:EM}

% \textbf{\cref{prop:bp}.}\textit{
% If \textsc{Refocus} selects a view non-deterministically from
% $\{ a\sto\mskip-2mu X, X\! \sto a, X\}_{X \in \X, a \in \partial X}$
% with $\varphi, \chi$ as above, and $\gamma=1$, then
% % with attention and control sets as above and
% % $\gamma=1$, then
% % \[ \Big\{
% %     \begin{pmatrix}
% %         % \chi = \infty \mathbbm 1[m_{X \sto a}]\\
% %         % \phi = \infty\mathbbm1 [\{ m_{b \sto X} \}_{b \in \partial X \setminus a} \cup \{m_{X \sto a}\}]
% %         C =  \{ m_{X \sto a} \}\\
% %         A = \{ m_{b \sto X} \}_{b \in \partial X \setminus a} \cup \{m_{X \sto a}\}
% %     \end{pmatrix}
% %     ~\Big|~ a \in \Ar, X \in \mat X_a \Big\},
% % \]
% the possible runs of
% {\sc LIR}$(
%     \dg M_\Phi, \Msg
%     + \dg B
%      )$
% are precisely those of BP for different message schedules.}
\recall{proposition:bp}
\begin{lproof} \label{proof:bp}
When $\gamma=1$, and $\alpha, \beta = 1$ for all of the input factors, then the optimal
distribution $\mu^*$ that realizes the infemum is just the product of factors. It follows that any distribution that has those marginals will minimize the observational inconsistency.

The different orders that the \eqref{eq:X->a}, and \eqref{eq:a->X} can be ordered
for different adjacent pairs $(a, X)$ correspond to both the message passing schedules, and to the possible view selections of LIR.
\end{lproof}


\recall{proposition:eumaxmax}
\begin{lproof}\label{proof:eumaxmax}
   Since the choice $a$ is deterministic, the value of $A$ is determined; likewise the value of $\Tru$ is also fixed. 
   Similarly, as $u$ is a deterministic function, the value of $U$ is determined according to $u$.
   Thus we need only consider distributions $\mu(S, O)$ in our minimization; the other variables can be found as a function of these.
   However, we also know that $\mu(O\mid S) = \tau(O|S,A{=}a)$ since it is given with high confidence. 
   Therefore it suffices to restrict our search to distributions over the variable $S$.
   
   To simplify notation, let 
   $ EU(s,a) := \Ex_{o \sim \tau (O|s,a)} [ u(o) ] + \log k$
   be the expected utility of an action, shifted by the constant $\log k$.
   Note that
   \[
       \log \frac{1}{b(\Tru{=}\trut\mid U{=}u(o))}
       = - \log (k \cdot \exp( u(o) ))
       = - u(o) - \log k,
   \]
   which in expectation over $\tau(O|s,a)$, is $ - EU(s,a)$. 
   % which happens to be equal to 
   % $\log \frac{1}{b(\Tru{=}\trut\mid U{=}u(o))} =  \log \exp u(o) $
   % We then calculate that
   With this in mind, we calculate:
   \begin{align*}
       \aar[\big]{\dg M_{p,\tau,u,b}+A{=}a}_{\gamma}
       &= \inf_{\mu(S)} 
           \beta_p \Ex_{s \sim \mu} \Big[ \log \frac {\mu(s)}{p(s)} 
           + \frac{\beta_b}{\beta_p} \Ex_{o\sim \tau|a,s}\Big[\log \frac1{b(\Tru{=}\trut\mid U{=}u(o))}\Big]
           \\
       &= \inf_{\mu(S)} 
           \beta_p \Ex_{s \sim \mu} \Big[ \log \frac {\mu(s)}{p(s)} 
           + \log \circ \exp \Big(- \frac{\beta_b}{\beta_p} EU(s,a) \Big)
           \\
       &= \inf_{\mu(S)} 
           \beta_p \Ex_{s \sim \mu} \Big[ \log \frac {\mu(s)}{1}\cdot\frac{\exp(-\frac{\beta_b}{\beta_p} EU(s,a))}{p(s)} \cdot \frac ZZ \Big]
           \\
       &= \inf_{\mu(S)} 
           \beta_p \Ex_{s \sim \mu} \Big[ \log \frac {\mu(s)}{1}\cdot\frac{Z}{p(s)\cdot \exp(\frac{\beta_b}{\beta_p} EU(s,a))} \cdot \frac 1Z \Big]
               \numberthis\label{eq:save-eu-calc}
   \end{align*}
   At this point, we can use the same trick used repeatedly in the proving results of \citet{one-true-loss}: take $Z$ to be the normalization constant needed to regard the middle fraction as the inverse of a probability distribution $\nu$. 
   Once we do so, we are left with an infimum over a KL divergence $\kldiv{\mu}{\nu}$ plus the expectation of a constant:
   \begin{align*}
       \aar[\big]{\dg M_{p,\tau,u,b}+A{=}a}_{\gamma}
           &= \beta_p \log \frac1 Z
           = - \log \sum_{s} p(s) \exp\Big( + \frac{\beta_b}{\beta_p} EU(s,a)\Big).
   \end{align*}
   We now look at the two extreme cases. 
   
   When $\beta_p \to \infty$, then
       the ratio $\frac{\beta_b}{\beta_p}$ becomes small, 
       and we can use the fact that $\exp( \epsilon) \approx 1+\epsilon$
       for small $\epsilon$
   to find that the inconsistency of interest is
   approximately 
   \begin{align*}
       \approx 
       - \beta_p \log \sum_{s} p(s) \Big[ 1 + \frac{\beta_b}{\beta_p} EU(s,a) \Big]
       = - \beta_p \log( 1 +  \frac{\beta_b}{\beta_p} EU(s,a) )
       \approx
       - \beta_b EU(s,a)
   \end{align*}
   Alternatively, more directly, when $\beta_p = \infty$, the optimal distribution must be $\mu = p$, and so the inconsistency is immediately $ - \Ex_{s \sim p} [ \beta_b EU(s,a)]$. 
   Either way, minimizing this quantity amounts to maximizing espected utility, as the two differ by a negative affine transformation.
   
   At the other extreme, when $\beta_b \to \infty$, 
   we can write our expression in terms of 
   $\mathrm{LSE} \{ x_1, \ldots, x_n \} := \log \sum_{i=1}^n \exp( x_i)$ (LogSumExp)
   which is a smooth approximation to a $\max$.  (In a moment, we will negate its arguments and the final output, using it as an approximation to a $\min$, instead.)
   Picking back up from \eqref{eq:save-eu-calc} and letting $t := \frac{\beta_b}{\beta_p}$, we find
   \begin{align*}
       \aar[\big]{\dg M_{p,\tau,u,b}+A{=}a}_{\gamma}
           &= - \beta_b \cdot \frac{-1}{t} \mathop{\mathrm{LSE}}\limits_{s \in \V S} \Big[-t \cdot \Big( \frac1t \log \frac1{p(s)} -  EU(s,a)\Big) \Big]. 
   \end{align*}
   Using the standard fact
   \unskip\footnote{
       Letting $m := \min_i x_i$, observe that, for all $t > 0$, we have
       $\exp(-tm) \le \sum_i \exp(-t x_i) \le n \exp(- tm)$.
       Apply a logarithm and multiply by $-\frac{1}{t}$ to get the promised result.}
   that 
   \[
       \min_{i \in [n]} x_i - \frac{1}{t} \log n \le 
           \frac{-1}{t} \mathop{\mathrm{LSE}}\limits_{i \in [n]} ( - t x_i ) < \min_{i \in [n]} x_i,
   \]
   we find that, in our case,
   \[
   M - \frac{1}{t} \log |\V S|
   ~~\le~~ \frac{1}{\beta_b} \aar[\big]{\dg M_{p,\tau,u,b}+A{=}a}_{\gamma}
   ~~\le~~ M
   \]
   where $M := \min_{s \in \V S}  ( -  EU(s,a)  + \frac{1}{t} \log \frac 1{p(s)} )$.
   In particular, when $\beta_b \to \infty$, meaning $t \to \infty$, the gap between the upper and lower bounds shrinks to zero,
   and the resulting inconsistency becomes
   $ - \min_{s \in \V S} ( - EU(s,a) ) = \max_{s \in \V S)} EU(s,a)$,
   proving the result.
\end{lproof}


\section{SCRATCH}

\subsection{A Justification}
The fact that one can describe many useful algorithms in a certain format is not, on its own, a very significant evidence of value. 
After all, once we know that all of mathematics can be described by sets, 

So we are trying to establish something deeper: that this is 
 

\end{document}
